\documentclass[12pt,a4paper]{book}

% ---------- Encoding & language ----------
\usepackage[utf8]{inputenc}
\usepackage[T1]{fontenc}
\usepackage[english]{babel}
\usepackage{lmodern}

% ---------- Layout ----------
\usepackage{geometry}
\geometry{margin=1in}
\usepackage{setspace}
\onehalfspacing

% ---------- Math & tables ----------
\usepackage{amsmath}
\usepackage{booktabs}
\usepackage{array}
\usepackage{multirow}

% ---------- Misc ----------
\usepackage{hyperref}
\usepackage{csquotes}

\title{Mastering English Tenses\\[0.5em]
\large From Absolute Beginner to Advanced User}
\author{Comprehensive Grammar Reference}
\date{\today}

\begin{document}
% Signature page
\thispagestyle{empty}
\begin{center}
\vspace*{3cm}
{\Large\itshape Made with love for my TCD classmates}\\[0.5em]
{\small by MUGANGU MUGISHO Benjamin}\\[2cm]
\end{center}

\selectlanguage{english}

\maketitle

\frontmatter

\tableofcontents

\chapter*{Preface}
\addcontentsline{toc}{chapter}{Preface}

This book is designed as a complete reference on English tenses. It aims
to help you understand, use, and master all the main tense forms in
English, from beginner level to advanced.

For each tense, you will find:

\begin{itemize}
  \item clear grammatical structure (affirmative, negative, interrogative);
  \item active and passive forms where appropriate;
  \item main uses, with explanations in simple English;
  \item many examples, from very simple to more advanced;
  \item common mistakes and important notes;
  \item summary tables to help you review.
\end{itemize}

You can read this book from the beginning if you are a beginner, or you
can use it as a reference and jump directly to the tense you need to
revise.

\chapter*{How to Use This Book}
\addcontentsline{toc}{chapter}{How to Use This Book}

\begin{itemize}
  \item If you are a beginner, focus first on the present, past, and
        future simple tenses, then gradually add continuous and perfect
        forms.
  \item If you are intermediate, pay special attention to the differences
        between similar tenses (for example, Present Perfect vs Past Simple).
  \item If you are advanced, study the perfect continuous tenses,
        passive forms, and conditional structures carefully.
  \item Revisit the summary tables in the last chapters to consolidate
        your knowledge.
\end{itemize}

\mainmatter

%=========================================================
\chapter{Foundations: Verbs, Time and Aspect}
%=========================================================

\section{The Core Verb Forms}

In English, most tenses are built from a small set of basic verb forms.
For each important verb you should know at least:

\begin{itemize}
  \item \textbf{Base form (infinitive)}: work, go, take, eat, see
  \item \textbf{Third person singular (present)}: works, goes, takes, eats, sees
  \item \textbf{Past simple}: worked, went, took, ate, saw
  \item \textbf{Past participle}: worked, gone, taken, eaten, seen
  \item \textbf{-ing form (present participle / gerund)}: working, going, taking, eating, seeing
\end{itemize}

These forms are used in all the tense patterns and in the passive voice.

\section{Auxiliary Verbs}

Auxiliary verbs are \enquote{helper} verbs used to form tenses, questions,
negatives, and passive forms.

\begin{itemize}
  \item \textbf{BE}: am, is, are, was, were, been, being
  \item \textbf{HAVE}: have, has, had
  \item \textbf{DO}: do, does, did
  \item \textbf{WILL}: will (for future)
\end{itemize}

Examples:

\begin{itemize}
  \item \enquote{I \textbf{am} working.} (Present Continuous)
  \item \enquote{She \textbf{has} finished.} (Present Perfect)
  \item \enquote{\textbf{Do} you like pizza?} (question with Present Simple)
  \item \enquote{The work \textbf{was} done yesterday.} (Past Simple passive)
\end{itemize}

\section{Time and Aspect}

English has three basic time references:

\begin{itemize}
  \item present
  \item past
  \item future
\end{itemize}

Each time reference can combine with four main aspects:

\begin{itemize}
  \item \textbf{Simple}: an action as a whole (I work, I worked)
  \item \textbf{Continuous / Progressive}: an action in progress (I am working)
  \item \textbf{Perfect}: a link between two points in time (I have worked)
  \item \textbf{Perfect Continuous}: link + duration (I have been working)
\end{itemize}

By combining time and aspect, we get the 12 main tenses that this book
will cover in detail.

\section{Overview of the 12 Main Tenses}

\begin{center}
\begin{tabular}{lll}
\toprule
           & \textbf{Simple}       & \textbf{Continuous (Progressive)} \\
\midrule
Present    & Present Simple        & Present Continuous                \\
Past       & Past Simple           & Past Continuous                   \\
Future     & Future Simple (will)  & Future Continuous                 \\
\midrule
           & \textbf{Perfect}      & \textbf{Perfect Continuous}      \\
\midrule
Present    & Present Perfect       & Present Perfect Continuous       \\
Past       & Past Perfect          & Past Perfect Continuous          \\
Future     & Future Perfect        & Future Perfect Continuous        \\
\bottomrule
\end{tabular}
\end{center}

Each of the next chapters is dedicated to one tense or group of tenses.

%=========================================================
\chapter{Present Simple}
%=========================================================

\section{Core Meaning}

The Present Simple is the tense of:

\begin{itemize}
  \item habits and routines;
  \item general truths and facts;
  \item permanent situations;
  \item official timetables and schedules;
  \item states (feelings, opinions, mental states).
\end{itemize}

It does not normally describe an action in progress \enquote{right now}.

\section{Forms}

\subsection{Affirmative}

\subsubsection*{Structure}

\begin{center}
Subject + base form \\
Subject + base form + \textbf{s} (3rd person singular)
\end{center}

\subsubsection*{Examples}

\begin{itemize}
  \item I work in a hospital.
  \item You play the guitar very well.
  \item We live in a small town.
  \item They speak three languages.
  \item He works in an office.
  \item She studies engineering.
  \item It rains a lot in this region.
\end{itemize}

\subsection{Negative}

\subsubsection*{Structure}

\begin{center}
Subject + do not (don’t) + base form \\
Subject + does not (doesn’t) + base form (3rd person singular)
\end{center}

\subsubsection*{Examples}

\begin{itemize}
  \item I do not (don’t) like coffee.
  \item We do not (don’t) watch TV on weekdays.
  \item He does not (doesn’t) eat meat.
  \item She does not (doesn’t) drive to work.
  \item It does not (doesn’t) look good.
\end{itemize}

\subsection{Interrogative}

\subsubsection*{Structure}

\begin{center}
Do + subject + base form? \\
Does + subject + base form? (3rd person singular)
\end{center}

\subsubsection*{Examples}

\begin{itemize}
  \item Do you speak English?
  \item Do they live near here?
  \item Does he work on Sundays?
  \item Does she like spicy food?
\end{itemize}

\subsection{Summary Table}

\begin{center}
\begin{tabular}{lll}
\toprule
Person & Affirmative & Negative / Question \\
\midrule
I      & I work.     & I don’t work. / Do I work? \\
You    & You work.   & You don’t work. / Do you work? \\
We     & We work.    & We don’t work. / Do we work? \\
They   & They work.  & They don’t work. / Do they work? \\
He     & He works.   & He doesn’t work. / Does he work? \\
She    & She works.  & She doesn’t work. / Does she work? \\
It     & It works.   & It doesn’t work. / Does it work? \\
\bottomrule
\end{tabular}
\end{center}

\section{Uses With Examples}

\subsection{Habits and Routines}

\begin{itemize}
  \item I get up at 6 a.m. every day.
  \item She usually goes to the gym after work.
  \item They often visit their grandparents on Sundays.
  \item He rarely eats dessert.
\end{itemize}

\subsection{General Truths and Facts}

\begin{itemize}
  \item Water boils at 100 degrees Celsius.
  \item The Earth orbits the Sun.
  \item Cats hate cold water.
\end{itemize}

\subsection{Permanent Situations}

\begin{itemize}
  \item She lives in New York.
  \item My parents own a small restaurant.
  \item The company produces software for schools.
\end{itemize}

\subsection{Timetables and Schedules}

\begin{itemize}
  \item The train leaves at 8:30.
  \item The shop opens at 9 a.m. and closes at 7 p.m.
  \item The exam starts at 2 o’clock.
\end{itemize}

\subsection{States (Stative Verbs)}

Some verbs usually describe states rather than actions. They are often
used in the Present Simple instead of the Present Continuous.

Common stative verbs:

\begin{itemize}
  \item \textbf{verbs of perception}: see, hear, smell, taste;
  \item \textbf{verbs of emotion}: like, love, hate, prefer;
  \item \textbf{verbs of thinking}: know, believe, remember, understand;
  \item others: want, need, seem, belong.
\end{itemize}

Examples:

\begin{itemize}
  \item I know the answer.
  \item She likes chocolate.
  \item They believe in hard work.
  \item This book belongs to me.
\end{itemize}

\section{Passive Voice in the Present Simple}

\subsection{Form}

\begin{center}
Subject + am / is / are + past participle
\end{center}

\subsection{Examples}

\begin{itemize}
  \item Active: They clean the office every day.\\
        Passive: The office \textbf{is cleaned} every day.
  \item Active: People speak English worldwide.\\
        Passive: English \textbf{is spoken} worldwide.
  \item Active: The company produces these devices in China.\\
        Passive: These devices \textbf{are produced} in China.
\end{itemize}

\section{Common Mistakes and Tips}

\begin{itemize}
  \item Do not forget the \textbf{-s} for he/she/it: \enquote{He works}, not \enquote{He work}.
  \item Use \textbf{do/does} only in negatives and questions, not in simple affirmative
        forms: \enquote{He works}, not \enquote{He does works}.
  \item Do not use the Present Simple for actions happening \enquote{right now};
        use the Present Continuous instead.
\end{itemize}

%=========================================================
\chapter{Present Continuous}
%=========================================================

\section{Core Meaning}

The Present Continuous (also called Present Progressive) focuses on:

\begin{itemize}
  \item actions happening \textbf{right now};
  \item temporary situations;
  \item changing situations;
  \item arrangements for the near future.
\end{itemize}

\section{Forms}

\subsection{Affirmative}

\begin{center}
Subject + am / is / are + verb-ing
\end{center}

Examples:

\begin{itemize}
  \item I am working.
  \item You are reading.
  \item He is studying English.
  \item She is cooking dinner.
  \item We are watching a movie.
  \item They are playing football.
\end{itemize}

\subsection{Negative}

\begin{center}
Subject + am / is / are + not + verb-ing
\end{center}

Examples:

\begin{itemize}
  \item I am not working right now.
  \item He is not (isn’t) listening.
  \item We are not (aren’t) doing our homework.
\end{itemize}

\subsection{Interrogative}

\begin{center}
Am / Is / Are + subject + verb-ing?
\end{center}

Examples:

\begin{itemize}
  \item Are you working?
  \item Is she sleeping?
  \item Are they coming to the party?
\end{itemize}

\subsection{Summary Table}

\begin{center}
\begin{tabular}{lll}
\toprule
Person & Affirmative & Negative / Question \\
\midrule
I   & I am working. & I am not working. / Am I working? \\
You & You are working. & You are not working. / Are you working? \\
He  & He is working. & He isn’t working. / Is he working? \\
She & She is working. & She isn’t working. / Is she working? \\
It  & It is working. & It isn’t working. / Is it working? \\
We  & We are working. & We aren’t working. / Are we working? \\
They& They are working. & They aren’t working. / Are they working? \\
\bottomrule
\end{tabular}
\end{center}

\section{Uses With Examples}

\subsection{Actions Happening Now}

\begin{itemize}
  \item I am talking to you right now.
  \item She is watching TV at the moment.
  \item They are having lunch.
\end{itemize}

\subsection{Temporary Situations}

\begin{itemize}
  \item He is living with his parents this month.
  \item I am working on a special project these days.
\end{itemize}

\subsection{Changing or Developing Situations}

\begin{itemize}
  \item More and more people are using smartphones.
  \item The climate is getting warmer.
  \item Prices are increasing slowly.
\end{itemize}

\subsection{Future Arrangements}

\begin{itemize}
  \item I am meeting my friend tomorrow at 5 p.m.
  \item We are flying to London next week.
\end{itemize}

\section{Passive Voice in the Present Continuous}

\subsection{Form}

\begin{center}
am / is / are + being + past participle
\end{center}

Examples:

\begin{itemize}
  \item Active: They are repairing the road.\\
        Passive: The road \textbf{is being repaired}.
  \item Active: Someone is painting the walls.\\
        Passive: The walls \textbf{are being painted}.
\end{itemize}

\section{Stative Verbs: Important Note}

Many stative verbs are not usually used in continuous forms, especially:

\begin{itemize}
  \item know, believe, understand, remember;
  \item like, love, hate, prefer;
  \item want, need, seem, belong.
\end{itemize}

We normally say:

\begin{itemize}
  \item I know him. (not: I am knowing him.)
  \item She likes this song. (not: She is liking this song.)
\end{itemize}

Some verbs can be stative or dynamic, with a change of meaning:

\begin{itemize}
  \item \textbf{think}:
    \begin{itemize}
      \item I think this is a good idea. (opinion, stative)
      \item I am thinking about my future. (mental activity in progress)
    \end{itemize}
  \item \textbf{have}:
    \begin{itemize}
      \item I have a car. (possession, stative)
      \item I am having lunch. (activity in progress)
    \end{itemize}
\end{itemize}

%=========================================================
\chapter{Present Perfect Simple}
%=========================================================

\section{Core Meaning}

The Present Perfect links the past and the present. It is used for:

\begin{itemize}
  \item life experience up to now;
  \item past actions with present result;
  \item recent events;
  \item actions that started in the past and continue now.
\end{itemize}

\section{Forms}

\subsection{Affirmative}

\begin{center}
Subject + have / has + past participle
\end{center}

Examples:

\begin{itemize}
  \item I have seen that movie.
  \item You have improved a lot.
  \item She has finished her homework.
  \item He has broken his leg.
  \item We have visited London many times.
  \item They have already left.
\end{itemize}

\subsection{Negative}

\begin{center}
Subject + have / has + not + past participle
\end{center}

Examples:

\begin{itemize}
  \item I have not (haven’t) seen him today.
  \item She has not (hasn’t) finished yet.
  \item They have not (haven’t) decided anything.
\end{itemize}

\subsection{Interrogative}

\begin{center}
Have / Has + subject + past participle?
\end{center}

Examples:

\begin{itemize}
  \item Have you eaten yet?
  \item Has he called you?
  \item How long have you lived here?
\end{itemize}

\section{Main Uses}

\subsection{Life Experience (Without Saying When)}

\begin{itemize}
  \item I have travelled to many countries.
  \item She has never flown in a plane.
  \item Have you ever tried sushi?
\end{itemize}

\subsection{Present Result of a Past Action}

\begin{itemize}
  \item She has broken her leg. (Her leg is broken now.)
  \item I have lost my keys. (I do not have them now.)
  \item Someone has stolen my phone. (It is missing now.)
\end{itemize}

\subsection{Recent Actions (just, already, yet)}

\begin{itemize}
  \item I have just finished my homework.
  \item They have already started the meeting.
  \item Have you finished yet?
\end{itemize}

\subsection{Actions Continuing Until Now (for, since)}

\begin{itemize}
  \item I have lived here for ten years.
  \item He has worked at this company since 2015.
  \item We have known each other since childhood.
\end{itemize}

\section{Present Perfect vs Past Simple}

\subsection{Past Simple: Finished Time}

Use the Past Simple with finished time expressions:

\begin{itemize}
  \item yesterday, last week, last year;
  \item in 2010, two days ago, when I was a child.
\end{itemize}

Examples:

\begin{itemize}
  \item I saw that movie yesterday.
  \item She visited London in 2018.
\end{itemize}

\subsection{Present Perfect: Unfinished Time or No Time Mentioned}

Examples:

\begin{itemize}
  \item I have seen that movie three times. (no specific time, experience)
  \item She has visited London several times. (experience)
  \item I have worked here since 2010. (period continues)
\end{itemize}

\section{Passive Voice in the Present Perfect}

\subsection{Form}

\begin{center}
have / has + been + past participle
\end{center}

Examples:

\begin{itemize}
  \item Active: They have repaired the car.\\
        Passive: The car \textbf{has been repaired}.
  \item Active: Someone has stolen my bike.\\
        Passive: My bike \textbf{has been stolen}.
\end{itemize}

\section{Common Mistakes and Tips}

\begin{itemize}
  \item Do not use the Present Perfect with finished time expressions:
        \enquote{I have seen him yesterday} (wrong) $\rightarrow$ \enquote{I saw him yesterday} (correct).
  \item For \enquote{how long} questions, prefer Present Perfect
        when the situation still continues:
        \enquote{How long have you lived here?}
\end{itemize}

%=========================================================
\chapter{Present Perfect Continuous}
%=========================================================

\section{Core Meaning}

The Present Perfect Continuous emphasizes:

\begin{itemize}
  \item the duration of an activity that started in the past and continues now;
  \item the activity itself (not just the result).
\end{itemize}

\section{Forms}

\subsection{Affirmative}

\begin{center}
Subject + have / has been + verb-ing
\end{center}

Examples:

\begin{itemize}
  \item I have been studying for two hours.
  \item She has been working all day.
  \item They have been waiting since 8 o’clock.
\end{itemize}

\subsection{Negative}

\begin{center}
Subject + have / has not been + verb-ing
\end{center}

Examples:

\begin{itemize}
  \item I haven’t been sleeping well recently.
  \item He hasn’t been studying enough this semester.
\end{itemize}

\subsection{Interrogative}

\begin{center}
Have / Has + subject + been + verb-ing?
\end{center}

Examples:

\begin{itemize}
  \item Have you been working hard?
  \item Has she been feeling better recently?
\end{itemize}

\section{Uses}

\subsection{Activities From the Past Until Now}

\begin{itemize}
  \item I have been learning English for five years.
  \item We have been working on this project since Monday.
\end{itemize}

\subsection{Activity Causing a Present Result}

\begin{itemize}
  \item She is tired because she has been running.
  \item His hands are dirty because he has been fixing the car.
\end{itemize}

\section{Present Perfect Simple vs Continuous}

\begin{itemize}
  \item Present Perfect Simple: focus on the result:
    \begin{itemize}
      \item I have read that book. (It is finished.)
    \end{itemize}
  \item Present Perfect Continuous: focus on the activity and duration:
    \begin{itemize}
      \item I have been reading that book. (Maybe not finished.)
    \end{itemize}
\end{itemize}

%=========================================================
\chapter{Past Simple}
%=========================================================

\section{Core Meaning}

The Past Simple describes:

\begin{itemize}
  \item completed actions in the past;
  \item actions with a definite time reference (even if not mentioned);
  \item sequences of actions in stories.
\end{itemize}

\section{Forms}

\subsection{Affirmative}

\begin{center}
Subject + past simple
\end{center}

Examples:

\begin{itemize}
  \item I visited my grandmother last weekend.
  \item She worked in a bank for five years.
  \item They moved to Canada in 2010.
\end{itemize}

\subsection{Negative}

\begin{center}
Subject + did not (didn’t) + base form
\end{center}

Examples:

\begin{itemize}
  \item I didn’t see him yesterday.
  \item We didn’t go to the party.
\end{itemize}

\subsection{Interrogative}

\begin{center}
Did + subject + base form?
\end{center}

Examples:

\begin{itemize}
  \item Did you enjoy the film?
  \item Did she pass the exam?
\end{itemize}

\section{Uses}

\subsection{Single Completed Actions}

\begin{itemize}
  \item I met him at a conference last year.
  \item She bought a new car yesterday.
\end{itemize}

\subsection{Series of Actions in a Story}

\begin{itemize}
  \item He got up, took a shower, and left the house.
  \item They arrived, checked in, and went to their rooms.
\end{itemize}

\subsection{Past Habits}

\begin{itemize}
  \item When I was a child, I played outside every day.
  \item We always visited our grandparents in the summer.
\end{itemize}

\section{Passive Voice in the Past Simple}

\begin{center}
was / were + past participle
\end{center}

Examples:

\begin{itemize}
  \item Active: They built this bridge in 1995.\\
        Passive: This bridge \textbf{was built} in 1995.
  \item Active: Somebody stole my bike.\\
        Passive: My bike \textbf{was stolen}.
\end{itemize}

%=========================================================
\chapter{Past Continuous}
%=========================================================

\section{Core Meaning}

The Past Continuous describes:

\begin{itemize}
  \item actions in progress at a specific time in the past;
  \item longer actions interrupted by shorter actions.
\end{itemize}

\section{Forms}

\subsection{Affirmative}

\begin{center}
Subject + was / were + verb-ing
\end{center}

Examples:

\begin{itemize}
  \item I was reading at 9 p.m. last night.
  \item They were playing football when it started to rain.
\end{itemize}

\subsection{Negative}

\begin{center}
Subject + was / were + not + verb-ing
\end{center}

Examples:

\begin{itemize}
  \item I wasn’t sleeping when you called.
  \item We weren’t watching TV at that time.
\end{itemize}

\subsection{Interrogative}

\begin{center}
Was / Were + subject + verb-ing?
\end{center}

Examples:

\begin{itemize}
  \item Were you studying when I phoned?
  \item Was she working yesterday afternoon?
\end{itemize}

\section{Uses}

\subsection{Background Actions}

\begin{itemize}
  \item At 8 p.m., I was having dinner.
  \item While they were driving home, it began to snow.
\end{itemize}

\subsection{Interrupted Actions}

\begin{itemize}
  \item I was watching TV when the phone rang.
  \item She was walking to work when she met an old friend.
\end{itemize}

%=========================================================
\chapter{Past Perfect Simple and Continuous}
%=========================================================

\section{Past Perfect Simple}

\subsection{Core Meaning}

The Past Perfect Simple is the \enquote{past of the past}. It shows that
one past action happened before another past action.

\subsection{Form}

\begin{center}
Subject + had + past participle
\end{center}

Examples:

\begin{itemize}
  \item When I arrived, they had already started the meeting.
  \item She had finished her homework before dinner.
\end{itemize}

\subsection{Passive}

\begin{center}
had + been + past participle
\end{center}

Example:

\begin{itemize}
  \item The work had been completed before the deadline.
\end{itemize}

\section{Past Perfect Continuous}

\subsection{Core Meaning}

Emphasizes duration before a past point:

\begin{itemize}
  \item He was tired because he had been working all day.
\end{itemize}

\subsection{Form}

\begin{center}
Subject + had been + verb-ing
\end{center}

Examples:

\begin{itemize}
  \item They had been waiting for two hours when the bus finally arrived.
\end{itemize}

%=========================================================
\chapter{Future Forms: Overview}
%=========================================================

\section{Main Future Forms}

English uses several structures to talk about the future:

\begin{itemize}
  \item will + base form (Future Simple);
  \item be going to + base form;
  \item Present Continuous with future meaning;
  \item Present Simple for timetables;
  \item Future Continuous, Future Perfect, Future Perfect Continuous.
\end{itemize}

Later chapters will describe each future tense in more detail.

%=========================================================
\chapter{Future Simple (will) and Going To}
%=========================================================

\section{Future Simple with \enquote{will}}

\subsection{Form}

\begin{center}
Subject + will + base form
\end{center}

Examples:

\begin{itemize}
  \item I will call you later.
  \item She will help you with your homework.
\end{itemize}

\subsection{Uses}

\begin{itemize}
  \item decisions made at the moment of speaking;
  \item predictions based on opinion;
  \item promises and offers.
\end{itemize}

\section{Be Going To}

\subsection{Form}

\begin{center}
Subject + am / is / are + going to + base form
\end{center}

Examples:

\begin{itemize}
  \item I am going to study tonight.
  \item They are going to visit their grandparents.
\end{itemize}

\subsection{Uses}

\begin{itemize}
  \item intentions decided before speaking;
  \item predictions based on present evidence:
    \begin{itemize}
      \item Look at those clouds. It is going to rain.
    \end{itemize}
\end{itemize}

%=========================================================
\chapter{Future Continuous, Perfect and Perfect Continuous}
%=========================================================

\section{Future Continuous}

\subsection{Form}

\begin{center}
Subject + will be + verb-ing
\end{center}

Example:

\begin{itemize}
  \item This time tomorrow, I will be flying to London.
\end{itemize}

\section{Future Perfect Simple}

\subsection{Form}

\begin{center}
Subject + will have + past participle
\end{center}

Example:

\begin{itemize}
  \item By Friday, I will have finished my report.
\end{itemize}

\section{Future Perfect Continuous}

\subsection{Form}

\begin{center}
Subject + will have been + verb-ing
\end{center}

Example:

\begin{itemize}
  \item By next year, I will have been working here for ten years.
\end{itemize}

%=========================================================
\chapter{Passive Voice in Detail}
%=========================================================

\section{Active vs Passive}

Active voice focuses on the doer of the action:

\begin{quote}
The teacher explains the lesson.
\end{quote}

Passive voice focuses on the receiver of the action:

\begin{quote}
The lesson is explained by the teacher.
\end{quote}

\section{General Pattern}

\begin{center}
BE (in the correct tense) + past participle
\end{center}

\section{Passive Forms Across Tenses}

\begin{center}
\begin{tabular}{lll}
\toprule
Tense & Active & Passive \\
\midrule
Present Simple & They clean the room. & The room is cleaned. \\
Present Continuous & They are cleaning the room. & The room is being cleaned. \\
Present Perfect & They have cleaned the room. & The room has been cleaned. \\
Past Simple & They cleaned the room. & The room was cleaned. \\
Past Continuous & They were cleaning the room. & The room was being cleaned. \\
Past Perfect & They had cleaned the room. & The room had been cleaned. \\
Future Simple & They will clean the room. & The room will be cleaned. \\
Future Perfect & They will have cleaned it. & It will have been cleaned. \\
\bottomrule
\end{tabular}
\end{center}

%=========================================================
\chapter{Conditionals and Time Clauses}
%=========================================================

\section{Zero, First, Second and Third Conditional}

\begin{center}
\begin{tabular}{lll}
\toprule
Type & If-clause & Main clause \\
\midrule
Zero  & Present Simple & Present Simple \\
First & Present Simple & will + base form \\
Second& Past Simple & would + base form \\
Third & Past Perfect & would have + past participle \\
\bottomrule
\end{tabular}
\end{center}

Examples:

\begin{itemize}
  \item If you heat water, it boils. (Zero: general truth)
  \item If it rains, I will stay home. (First: real future)
  \item If I had more time, I would travel more. (Second: unreal present)
  \item If I had studied, I would have passed the exam. (Third: unreal past)
\end{itemize}

\section{Time Clauses with When, After, Before, Until}

In time clauses about the future, English often uses the Present Simple:

\begin{itemize}
  \item When I arrive, I will call you.
  \item As soon as she finishes, she will send the report.
\end{itemize}

%=========================================================
\chapter{Summary Tables and Study Plan}
%=========================================================

\section{Summary of Simple Tenses}

\begin{center}
\begin{tabular}{lll}
\toprule
Tense & Form & Main Use \\
\midrule
Present Simple & do/does + base & habits, general truths \\
Past Simple & did + base & completed past actions \\
Future Simple & will + base & decisions, opinions, promises \\
\bottomrule
\end{tabular}
\end{center}

\section{Summary of Continuous Tenses}

\begin{center}
\begin{tabular}{lll}
\toprule
Tense & Form & Main Idea \\
\midrule
Present Continuous & am/is/are + -ing & action now / temporary \\
Past Continuous & was/were + -ing & action in progress in the past \\
Future Continuous & will be + -ing & action in progress in the future \\
\bottomrule
\end{tabular}
\end{center}

\section{Summary of Perfect Tenses}

\begin{center}
\begin{tabular}{lll}
\toprule
Tense & Form & Time Link \\
\midrule
Present Perfect & have/has + pp & past linked to present \\
Past Perfect & had + pp & past before past \\
Future Perfect & will have + pp & finished before a future time \\
\bottomrule
\end{tabular}
\end{center}

\section{Study Plan}

\begin{itemize}
  \item \textbf{Beginner}: Present Simple, Present Continuous, Past Simple,
        Future with \enquote{will} and \enquote{going to}.
  \item \textbf{Intermediate}: all present and past tenses, basic conditionals.
  \item \textbf{Advanced}: all perfect and perfect continuous tenses,
        passive forms, conditionals 0, 1, 2, 3.
\end{itemize}

%=========================================================
\chapter{Word Stress: The Rhythm of English}
%=========================================================

\section{Importance of Word Stress}

In English, we do not say each syllable with the same force. In one word,
there is always one syllable that is louder, longer, and higher in pitch
than the others. This is called \textbf{word stress}.

Understanding stress is crucial because:
\begin{itemize}
  \item It helps with listening comprehension;
  \item It makes your speech sound more natural;
  \item It can change the meaning of a word (e.g., \textit{REcord} vs \textit{reCORD}).
\end{itemize}

\section{Basic Rules of Word Stress}

While there are many exceptions, these basic rules help in most cases:

\begin{enumerate}
  \item \textbf{One word has only one main stress.}
  \item \textbf{We only stress vowels, not consonants.}
\end{enumerate}

\section{Stress According to Word Class}

\subsection{Two-Syllable Nouns and Adjectives}
Most two-syllable nouns and adjectives have the stress on the \textbf{first} syllable.

\begin{itemize}
  \item \textbf{TA}-ble, \textbf{PRES}-ent, \textbf{CLEV}-er, \textbf{HAP}-py.
\end{itemize}

\subsection{Two-Syllable Verbs}
Most two-syllable verbs have the stress on the \textbf{second} syllable.

\begin{itemize}
  \item de-\textbf{CIDE}, be-\textbf{GIN}, pre-\textbf{SENT}, re-\textbf{CORD}.
\end{itemize}

\section{Noun-Verb Pairs (Heteronyms)}

Some words change their stress depending on whether they are used as a
noun or a verb.

\begin{center}
\begin{tabular}{lll}
\toprule
Word & Noun (Stress on 1st) & Verb (Stress on 2nd) \\
\midrule
Object  & \textbf{OB}-ject (a thing) & ob-\textbf{JECT} (to disagree) \\
Present & \textbf{PRES}-ent (a gift) & pre-\textbf{SENT} (to give) \\
Record  & \textbf{RE}-cord (a disk)  & re-\textbf{CORD} (to store sound) \\
Desert  & \textbf{DE}-sert (sand)    & de-\textbf{SERT} (to abandon) \\
\bottomrule
\end{tabular}
\end{center}

\section{Stress With Suffixes}

\subsection{Suffixes that do not change stress}
Most common suffixes (\textit{-ed, -ing, -er, -es, -ly, -ness}) do not
change the stress of the base word.
\begin{itemize}
  \item \textbf{HAP}-py $\rightarrow$ \textbf{HAP}-pi-ly.
  \item \textbf{WORK} $\rightarrow$ \textbf{WORK}-er.
\end{itemize}

\subsection{Suffixes that attract stress}
Some suffixes always take the stress themselves.
\begin{itemize}
  \item \textit{-ee}: em-ploy-\textbf{EE}, ref-u-\textbf{GEE}.
  \item \textit{-ese}: Chi-\textbf{NESE}, Por-tu-\textbf{GUESE}.
  \item \textit{-ique}: u-\textbf{NIQUE}, an-\textbf{TIQUE}.
\end{itemize}

\section{Compound Words}

\begin{itemize}
  \item \textbf{Compound Nouns}: Stress is usually on the \textbf{first} word.\\
    Example: \textbf{GREEN}-house, \textbf{BLACK}-bird.
  \item \textbf{Compound Verbs/Adjectives}: Stress is usually on the \textbf{second} word.\\
    Example: to over-\textbf{FLOW}, old-\textbf{FASH}-ioned.
\end{itemize}

%=========================================================
\chapter{Storytelling: Connecting the Tenses}
%=========================================================

\section{The Narrative Framework}

When we tell a story, we use a combination of different past tenses to
organise information and guide the listener. This is often called
\textbf{Narrative Tenses}.

\section{The Roles of Different Tenses}

\begin{itemize}
  \item \textbf{Past Simple}: For the main events of the story (the \enquote{backbone}).
  \item \textbf{Past Continuous}: For background scenes and actions in progress.
  \item \textbf{Past Perfect}: For events that happened \textit{before} the main story starts.
  \item \textbf{Present Simple/Continuous}: Can be used for \enquote{dramatic effect}
        (the historic present).
\end{itemize}

\section{Structuring Your Story}

A good story often follows this structure:

\begin{enumerate}
    \item \textbf{The Setting}: Use Past Continuous and Past Perfect.
    \item \textbf{The Inciting Incident}: Use Past Simple.
    \item \textbf{The Action}: Use Past Simple.
    \item \textbf{The Resolution}: Use Past Simple.
\end{enumerate}

\section{Linkers and Transitions}

To keep the story flowing, use time expressions:

\begin{itemize}
  \item \textbf{To start}: Once upon a time, It all began when...
  \item \textbf{To show sequence}: Then, after that, subsequently, later on.
  \item \textbf{To show simultaneous actions}: While, as, meanwhile.
  \item \textbf{To show surprise}: Suddenly, all of a sudden, out of the blue.
\end{itemize}

\section{A Practical Example: The Lost Cabin}

\begin{displayquote}
It \textbf{was raining} heavily (Past Continuous - background). I \textbf{had been walking}
for three hours (Past Perfect Continuous - duration before) and I \textbf{had lost}
my map (Past Perfect - event before).

Suddenly, I \textbf{saw} a small cabin (Past Simple - main event). I \textbf{approached}
the door and \textbf{knocked} (Past Simple - sequence). A man \textbf{opened} it.
He \textbf{was wearing} a strange hat...
\end{displayquote}

\section{Practice Table: Transforming Narrative}

\begin{center}
\begin{tabular}{ll}
\toprule
Basic Fact (Past Simple) & Narrative Enrichment \\
\midrule
I saw a dog. & I was walking home when I saw a dog. \\
The bus left. & I arrived at the station, but the bus had already left. \\
It rained. & It had been raining all day, so the ground was wet. \\
\bottomrule
\end{tabular}
\end{center}

\backmatter

\chapter*{Conclusion}
\addcontentsline{toc}{chapter}{Conclusion}

This book has presented the main English tenses in a systematic way.
To truly master them, you should now practise:

\begin{itemize}
  \item by writing your own sentences for each tense;
  \item by transforming sentences from one tense to another;
  \item by reading real texts and identifying the tenses used;
  \item by speaking and listening regularly in English.
\end{itemize}

Over time, accurate use of tenses will become natural and intuitive.

\end{document}
