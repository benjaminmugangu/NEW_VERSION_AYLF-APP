\documentclass[12pt,a4paper]{book}

% ---------- Encoding & language ----------
\usepackage[utf8]{inputenc}
\usepackage[T1]{fontenc}
\usepackage[english]{babel}
\usepackage{lmodern}

% ---------- Layout ----------
\usepackage{geometry}
\geometry{margin=1in}
\usepackage{setspace}
\onehalfspacing

% ---------- Math & tables ----------
\usepackage{amsmath}
\usepackage{booktabs}
\usepackage{array}
\usepackage{multirow}

% ---------- Misc ----------
\usepackage{hyperref}
\usepackage{csquotes}

\title{Mastering English Tenses\\[0.5em]
\large From Absolute Beginner to Advanced User}
\author{Comprehensive Grammar Reference}
\date{\today}

\begin{document}
% Signature page
\thispagestyle{empty}
\begin{center}
\vspace*{3cm}
{\Large\itshape Made with love for my TCD classmates}\\[0.5em]
{\small by MUGANGU MUGISHO Benjamin}\\[2cm]
\end{center}

\selectlanguage{english}

\maketitle

\frontmatter

\tableofcontents

\chapter*{Preface}
\addcontentsline{toc}{chapter}{Preface}

This book is designed as a complete reference on English tenses. It aims
to help you understand, use, and master all the main tense forms in
English, from beginner level to advanced.

For each tense, you will find:

\begin{itemize}
  \item clear grammatical structure (affirmative, negative, interrogative);
  \item active and passive forms where appropriate;
  \item main uses, with explanations in simple English;
  \item many examples, from very simple to more advanced;
  \item common mistakes and important notes;
  \item summary tables to help you review.
\end{itemize}

You can read this book from the beginning if you are a beginner, or you
can use it as a reference and jump directly to the tense you need to
revise.

\chapter*{How to Use This Book}
\addcontentsline{toc}{chapter}{How to Use This Book}

\begin{itemize}
  \item If you are a beginner, focus first on the present, past, and
        future simple tenses, then gradually add continuous and perfect
        forms.
  \item If you are intermediate, pay special attention to the differences
        between similar tenses (for example, Present Perfect vs Past Simple).
  \item If you are advanced, study the perfect continuous tenses,
        passive forms, and conditional structures carefully.
  \item Revisit the summary tables in the last chapters to consolidate
        your knowledge.
\end{itemize}

\mainmatter

%=========================================================
\chapter{Foundations: Verbs, Time and Aspect}
%=========================================================

\section{The Core Verb Forms}

In English, most tenses are built from a small set of basic verb forms.
For each important verb you should know at least:

\begin{itemize}
  \item \textbf{Base form (infinitive)}: work, go, take, eat, see
  \item \textbf{Third person singular (present)}: works, goes, takes, eats, sees
  \item \textbf{Past simple}: worked, went, took, ate, saw
  \item \textbf{Past participle}: worked, gone, taken, eaten, seen
  \item \textbf{-ing form (present participle / gerund)}: working, going, taking, eating, seeing
\end{itemize}

These forms are used in all the tense patterns and in the passive voice.

\section{Auxiliary Verbs}

Auxiliary verbs are \enquote{helper} verbs used to form tenses, questions,
negatives, and passive forms.

\begin{itemize}
  \item \textbf{BE}: am, is, are, was, were, been, being
  \item \textbf{HAVE}: have, has, had
  \item \textbf{DO}: do, does, did
  \item \textbf{WILL}: will (for future)
\end{itemize}

Examples:

\begin{itemize}
  \item \enquote{I \textbf{am} working.} (Present Continuous)
  \item \enquote{She \textbf{has} finished.} (Present Perfect)
  \item \enquote{\textbf{Do} you like pizza?} (question with Present Simple)
  \item \enquote{The work \textbf{was} done yesterday.} (Past Simple passive)
\end{itemize}

\section{Time and Aspect}

English has three basic time references:

\begin{itemize}
  \item present
  \item past
  \item future
\end{itemize}

Each time reference can combine with four main aspects:

\begin{itemize}
  \item \textbf{Simple}: an action as a whole (I work, I worked)
  \item \textbf{Continuous / Progressive}: an action in progress (I am working)
  \item \textbf{Perfect}: a link between two points in time (I have worked)
  \item \textbf{Perfect Continuous}: link + duration (I have been working)
\end{itemize}

By combining time and aspect, we get the 12 main tenses that this book
will cover in detail.

\section{Overview of the 12 Main Tenses}

\begin{center}
\begin{tabular}{lll}
\toprule
           & \textbf{Simple}       & \textbf{Continuous (Progressive)} \\
\midrule
Present    & Present Simple        & Present Continuous                \\
Past       & Past Simple           & Past Continuous                   \\
Future     & Future Simple (will)  & Future Continuous                 \\
\midrule
           & \textbf{Perfect}      & \textbf{Perfect Continuous}      \\
\midrule
Present    & Present Perfect       & Present Perfect Continuous       \\
Past       & Past Perfect          & Past Perfect Continuous          \\
Future     & Future Perfect        & Future Perfect Continuous        \\
\bottomrule
\end{tabular}
\end{center}

Each of the next chapters is dedicated to one tense or group of tenses.

%=========================================================
\chapter{Present Simple}
%=========================================================

\section{Core Meaning}

The Present Simple is the tense of:

\begin{itemize}
  \item habits and routines;
  \item general truths and facts;
  \item permanent situations;
  \item official timetables and schedules;
  \item states (feelings, opinions, mental states).
\end{itemize}

It does not normally describe an action in progress \enquote{right now}.

\section{Forms}

\subsection{Affirmative}

\subsubsection*{Structure}

\begin{center}
Subject + base form \\
Subject + base form + \textbf{s} (3rd person singular)
\end{center}

\subsubsection*{Examples}

\begin{itemize}
  \item I work in a hospital.
  \item You play the guitar very well.
  \item We live in a small town.
  \item They speak three languages.
  \item He works in an office.
  \item She studies engineering.
  \item It rains a lot in this region.
\end{itemize}

\subsection{Negative}

\subsubsection*{Structure}

\begin{center}
Subject + do not (don’t) + base form \\
Subject + does not (doesn’t) + base form (3rd person singular)
\end{center}

\subsubsection*{Examples}

\begin{itemize}
  \item I do not (don’t) like coffee.
  \item We do not (don’t) watch TV on weekdays.
  \item He does not (doesn’t) eat meat.
  \item She does not (doesn’t) drive to work.
  \item It does not (doesn’t) look good.
\end{itemize}

\subsection{Interrogative}

\subsubsection*{Structure}

\begin{center}
Do + subject + base form? \\
Does + subject + base form? (3rd person singular)
\end{center}

\subsubsection*{Examples}

\begin{itemize}
  \item Do you speak English?
  \item Do they live near here?
  \item Does he work on Sundays?
  \item Does she like spicy food?
\end{itemize}

\subsection{Summary Table}

\begin{center}
\begin{tabular}{lll}
\toprule
Person & Affirmative & Negative / Question \\
\midrule
I      & I work.     & I don’t work. / Do I work? \\
You    & You work.   & You don’t work. / Do you work? \\
We     & We work.    & We don’t work. / Do we work? \\
They   & They work.  & They don’t work. / Do they work? \\
He     & He works.   & He doesn’t work. / Does he work? \\
She    & She works.  & She doesn’t work. / Does she work? \\
It     & It works.   & It doesn’t work. / Does it work? \\
\bottomrule
\end{tabular}
\end{center}

\section{Uses With Examples}

\subsection{Habits and Routines}

\begin{itemize}
  \item I get up at 6 a.m. every day.
  \item She usually goes to the gym after work.
  \item They often visit their grandparents on Sundays.
  \item He rarely eats dessert.
\end{itemize}

\subsection{General Truths and Facts}

\begin{itemize}
  \item Water boils at 100 degrees Celsius.
  \item The Earth orbits the Sun.
  \item Cats hate cold water.
\end{itemize}

\subsection{Permanent Situations}

\begin{itemize}
  \item She lives in New York.
  \item My parents own a small restaurant.
  \item The company produces software for schools.
\end{itemize}

\subsection{Timetables and Schedules}

\begin{itemize}
  \item The train leaves at 8:30.
  \item The shop opens at 9 a.m. and closes at 7 p.m.
  \item The exam starts at 2 o’clock.
\end{itemize}

\subsection{States (Stative Verbs)}

Some verbs usually describe states rather than actions. They are often
used in the Present Simple instead of the Present Continuous.

Common stative verbs:

\begin{itemize}
  \item \textbf{verbs of perception}: see, hear, smell, taste;
  \item \textbf{verbs of emotion}: like, love, hate, prefer;
  \item \textbf{verbs of thinking}: know, believe, remember, understand;
  \item others: want, need, seem, belong.
\end{itemize}

Examples:

\begin{itemize}
  \item I know the answer.
  \item She likes chocolate.
  \item They believe in hard work.
  \item This book belongs to me.
\end{itemize}

\section{Passive Voice in the Present Simple}

\subsection{Form}

\begin{center}
Subject + am / is / are + past participle
\end{center}

\subsection{Examples}

\begin{itemize}
  \item Active: They clean the office every day.\\
        Passive: The office \textbf{is cleaned} every day.
  \item Active: People speak English worldwide.\\
        Passive: English \textbf{is spoken} worldwide.
  \item Active: The company produces these devices in China.\\
        Passive: These devices \textbf{are produced} in China.
\end{itemize}

\section{Common Mistakes and Tips}

\begin{itemize}
  \item Do not forget the \textbf{-s} for he/she/it: \enquote{He works}, not \enquote{He work}.
  \item Use \textbf{do/does} only in negatives and questions, not in simple affirmative
        forms: \enquote{He works}, not \enquote{He does works}.
  \item Do not use the Present Simple for actions happening \enquote{right now};
        use the Present Continuous instead.
\end{itemize}

%=========================================================
\chapter{Present Continuous}
%=========================================================

\section{Core Meaning}

The Present Continuous (also called Present Progressive) focuses on:

\begin{itemize}
  \item actions happening \textbf{right now};
  \item temporary situations;
  \item changing situations;
  \item arrangements for the near future.
\end{itemize}

\section{Forms}

\subsection{Affirmative}

\begin{center}
Subject + am / is / are + verb-ing
\end{center}

Examples:

\begin{itemize}
  \item I am working.
  \item You are reading.
  \item He is studying English.
  \item She is cooking dinner.
  \item We are watching a movie.
  \item They are playing football.
\end{itemize}

\subsection{Negative}

\begin{center}
Subject + am / is / are + not + verb-ing
\end{center}

Examples:

\begin{itemize}
  \item I am not working right now.
  \item He is not (isn’t) listening.
  \item We are not (aren’t) doing our homework.
\end{itemize}

\subsection{Interrogative}

\begin{center}
Am / Is / Are + subject + verb-ing?
\end{center}

Examples:

\begin{itemize}
  \item Are you working?
  \item Is she sleeping?
  \item Are they coming to the party?
\end{itemize}

\subsection{Summary Table}

\begin{center}
\begin{tabular}{lll}
\toprule
Person & Affirmative & Negative / Question \\
\midrule
I   & I am working. & I am not working. / Am I working? \\
You & You are working. & You are not working. / Are you working? \\
He  & He is working. & He isn’t working. / Is he working? \\
She & She is working. & She isn’t working. / Is she working? \\
It  & It is working. & It isn’t working. / Is it working? \\
We  & We are working. & We aren’t working. / Are we working? \\
They& They are working. & They aren’t working. / Are they working? \\
\bottomrule
\end{tabular}
\end{center}

\section{Uses With Examples}

\subsection{Actions Happening Now}

\begin{itemize}
  \item I am talking to you right now.
  \item She is watching TV at the moment.
  \item They are having lunch.
\end{itemize}

\subsection{Temporary Situations}

\begin{itemize}
  \item He is living with his parents this month.
  \item I am working on a special project these days.
\end{itemize}

\subsection{Changing or Developing Situations}

\begin{itemize}
  \item More and more people are using smartphones.
  \item The climate is getting warmer.
  \item Prices are increasing slowly.
\end{itemize}

\subsection{Future Arrangements}

\begin{itemize}
  \item I am meeting my friend tomorrow at 5 p.m.
  \item We are flying to London next week.
\end{itemize}

\section{Passive Voice in the Present Continuous}

\subsection{Form}

\begin{center}
am / is / are + being + past participle
\end{center}

Examples:

\begin{itemize}
  \item Active: They are repairing the road.\\
        Passive: The road \textbf{is being repaired}.
  \item Active: Someone is painting the walls.\\
        Passive: The walls \textbf{are being painted}.
\end{itemize}

\section{Stative Verbs: Important Note}

Many stative verbs are not usually used in continuous forms, especially:

\begin{itemize}
  \item know, believe, understand, remember;
  \item like, love, hate, prefer;
  \item want, need, seem, belong.
\end{itemize}

We normally say:

\begin{itemize}
  \item I know him. (not: I am knowing him.)
  \item She likes this song. (not: She is liking this song.)
\end{itemize}

Some verbs can be stative or dynamic, with a change of meaning:

\begin{itemize}
  \item \textbf{think}:
    \begin{itemize}
      \item I think this is a good idea. (opinion, stative)
      \item I am thinking about my future. (mental activity in progress)
    \end{itemize}
  \item \textbf{have}:
    \begin{itemize}
      \item I have a car. (possession, stative)
      \item I am having lunch. (activity in progress)
    \end{itemize}
\end{itemize}

%=========================================================
\chapter{Present Perfect Simple}
%=========================================================

\section{Core Meaning}

The Present Perfect links the past and the present. It is used for:

\begin{itemize}
  \item life experience up to now;
  \item past actions with present result;
  \item recent events;
  \item actions that started in the past and continue now.
\end{itemize}

\section{Forms}

\subsection{Affirmative}

\begin{center}
Subject + have / has + past participle
\end{center}

Examples:

\begin{itemize}
  \item I have seen that movie.
  \item You have improved a lot.
  \item She has finished her homework.
  \item He has broken his leg.
  \item We have visited London many times.
  \item They have already left.
\end{itemize}

\subsection{Negative}

\begin{center}
Subject + have / has + not + past participle
\end{center}

Examples:

\begin{itemize}
  \item I have not (haven’t) seen him today.
  \item She has not (hasn’t) finished yet.
  \item They have not (haven’t) decided anything.
\end{itemize}

\subsection{Interrogative}

\begin{center}
Have / Has + subject + past participle?
\end{center}

Examples:

\begin{itemize}
  \item Have you eaten yet?
  \item Has he called you?
  \item How long have you lived here?
\end{itemize}

\section{Main Uses}

\subsection{Life Experience (Without Saying When)}

\begin{itemize}
  \item I have travelled to many countries.
  \item She has never flown in a plane.
  \item Have you ever tried sushi?
\end{itemize}

\subsection{Present Result of a Past Action}

\begin{itemize}
  \item She has broken her leg. (Her leg is broken now.)
  \item I have lost my keys. (I do not have them now.)
  \item Someone has stolen my phone. (It is missing now.)
\end{itemize}

\subsection{Recent Actions (just, already, yet)}

\begin{itemize}
  \item I have just finished my homework.
  \item They have already started the meeting.
  \item Have you finished yet?
\end{itemize}

\subsection{Actions Continuing Until Now (for, since)}

\begin{itemize}
  \item I have lived here for ten years.
  \item He has worked at this company since 2015.
  \item We have known each other since childhood.
\end{itemize}

\section{Present Perfect vs Past Simple}

\subsection{Past Simple: Finished Time}

Use the Past Simple with finished time expressions:

\begin{itemize}
  \item yesterday, last week, last year;
  \item in 2010, two days ago, when I was a child.
\end{itemize}

Examples:

\begin{itemize}
  \item I saw that movie yesterday.
  \item She visited London in 2018.
\end{itemize}

\subsection{Present Perfect: Unfinished Time or No Time Mentioned}

Examples:

\begin{itemize}
  \item I have seen that movie three times. (no specific time, experience)
  \item She has visited London several times. (experience)
  \item I have worked here since 2010. (period continues)
\end{itemize}

\section{Passive Voice in the Present Perfect}

\subsection{Form}

\begin{center}
have / has + been + past participle
\end{center}

Examples:

\begin{itemize}
  \item Active: They have repaired the car.\\
        Passive: The car \textbf{has been repaired}.
  \item Active: Someone has stolen my bike.\\
        Passive: My bike \textbf{has been stolen}.
\end{itemize}

\section{Common Mistakes and Tips}

\begin{itemize}
  \item Do not use the Present Perfect with finished time expressions:
        \enquote{I have seen him yesterday} (wrong) $\rightarrow$ \enquote{I saw him yesterday} (correct).
  \item For \enquote{how long} questions, prefer Present Perfect
        when the situation still continues:
        \enquote{How long have you lived here?}
\end{itemize}

%=========================================================
\chapter{Present Perfect Continuous}
%=========================================================

\section{Core Meaning}

The Present Perfect Continuous emphasizes:

\begin{itemize}
  \item the duration of an activity that started in the past and continues now;
  \item the activity itself (not just the result).
\end{itemize}

\section{Forms}

\subsection{Affirmative}

\begin{center}
Subject + have / has been + verb-ing
\end{center}

Examples:

\begin{itemize}
  \item I have been studying for two hours.
  \item She has been working all day.
  \item They have been waiting since 8 o’clock.
\end{itemize}

\subsection{Negative}

\begin{center}
Subject + have / has not been + verb-ing
\end{center}

Examples:

\begin{itemize}
  \item I haven’t been sleeping well recently.
  \item He hasn’t been studying enough this semester.
\end{itemize}

\subsection{Interrogative}

\begin{center}
Have / Has + subject + been + verb-ing?
\end{center}

Examples:

\begin{itemize}
  \item Have you been working hard?
  \item Has she been feeling better recently?
\end{itemize}

\section{Uses}

\subsection{Activities From the Past Until Now}

\begin{itemize}
  \item I have been learning English for five years.
  \item We have been working on this project since Monday.
\end{itemize}

\subsection{Activity Causing a Present Result}

\begin{itemize}
  \item She is tired because she has been running.
  \item His hands are dirty because he has been fixing the car.
\end{itemize}

\section{Present Perfect Simple vs Continuous}

\begin{itemize}
  \item Present Perfect Simple: focus on the result:
    \begin{itemize}
      \item I have read that book. (It is finished.)
    \end{itemize}
  \item Present Perfect Continuous: focus on the activity and duration:
    \begin{itemize}
      \item I have been reading that book. (Maybe not finished.)
    \end{itemize}
\end{itemize}

%=========================================================
\chapter{Past Simple}
%=========================================================

\section{Core Meaning}

The Past Simple describes:

\begin{itemize}
  \item completed actions in the past;
  \item actions with a definite time reference (even if not mentioned);
  \item sequences of actions in stories.
\end{itemize}

\section{Forms}

\subsection{Affirmative}

\begin{center}
Subject + past simple
\end{center}

Examples:

\begin{itemize}
  \item I visited my grandmother last weekend.
  \item She worked in a bank for five years.
  \item They moved to Canada in 2010.
\end{itemize}

\subsection{Negative}

\begin{center}
Subject + did not (didn’t) + base form
\end{center}

Examples:

\begin{itemize}
  \item I didn’t see him yesterday.
  \item We didn’t go to the party.
\end{itemize}

\subsection{Interrogative}

\begin{center}
Did + subject + base form?
\end{center}

Examples:

\begin{itemize}
  \item Did you enjoy the film?
  \item Did she pass the exam?
\end{itemize}

\section{Uses}

\subsection{Single Completed Actions}

\begin{itemize}
  \item I met him at a conference last year.
  \item She bought a new car yesterday.
\end{itemize}

\subsection{Series of Actions in a Story}

\begin{itemize}
  \item He got up, took a shower, and left the house.
  \item They arrived, checked in, and went to their rooms.
\end{itemize}

\subsection{Past Habits}

\begin{itemize}
  \item When I was a child, I played outside every day.
  \item We always visited our grandparents in the summer.
\end{itemize}

\section{Passive Voice in the Past Simple}

\begin{center}
was / were + past participle
\end{center}

Examples:

\begin{itemize}
  \item Active: They built this bridge in 1995.\\
        Passive: This bridge \textbf{was built} in 1995.
  \item Active: Somebody stole my bike.\\
        Passive: My bike \textbf{was stolen}.
\end{itemize}

%=========================================================
\chapter{Past Continuous}
%=========================================================

\section{Core Meaning}

The Past Continuous describes:

\begin{itemize}
  \item actions in progress at a specific time in the past;
  \item longer actions interrupted by shorter actions.
\end{itemize}

\section{Forms}

\subsection{Affirmative}

\begin{center}
Subject + was / were + verb-ing
\end{center}

Examples:

\begin{itemize}
  \item I was reading at 9 p.m. last night.
  \item They were playing football when it started to rain.
\end{itemize}

\subsection{Negative}

\begin{center}
Subject + was / were + not + verb-ing
\end{center}

Examples:

\begin{itemize}
  \item I wasn’t sleeping when you called.
  \item We weren’t watching TV at that time.
\end{itemize}

\subsection{Interrogative}

\begin{center}
Was / Were + subject + verb-ing?
\end{center}

Examples:

\begin{itemize}
  \item Were you studying when I phoned?
  \item Was she working yesterday afternoon?
\end{itemize}

\section{Uses}

\subsection{Background Actions}

\begin{itemize}
  \item At 8 p.m., I was having dinner.
  \item While they were driving home, it began to snow.
\end{itemize}

\subsection{Interrupted Actions}

\begin{itemize}
  \item I was watching TV when the phone rang.
  \item She was walking to work when she met an old friend.
\end{itemize}

%=========================================================
\chapter{Past Perfect Simple and Continuous}
%=========================================================

\section{Past Perfect Simple}

\subsection{Core Meaning}

The Past Perfect Simple is the \enquote{past of the past}. It shows that
one past action happened before another past action.

\subsection{Form}

\begin{center}
Subject + had + past participle
\end{center}

Examples:

\begin{itemize}
  \item When I arrived, they had already started the meeting.
  \item She had finished her homework before dinner.
\end{itemize}

\subsection{Passive}

\begin{center}
had + been + past participle
\end{center}

Example:

\begin{itemize}
  \item The work had been completed before the deadline.
\end{itemize}

\section{Past Perfect Continuous}

\subsection{Core Meaning}

Emphasizes duration before a past point:

\begin{itemize}
  \item He was tired because he had been working all day.
\end{itemize}

\subsection{Form}

\begin{center}
Subject + had been + verb-ing
\end{center}

Examples:

\begin{itemize}
  \item They had been waiting for two hours when the bus finally arrived.
\end{itemize}

%=========================================================
\chapter{Future Forms: Overview}
%=========================================================

\section{Main Future Forms}

English uses several structures to talk about the future:

\begin{itemize}
  \item will + base form (Future Simple);
  \item be going to + base form;
  \item Present Continuous with future meaning;
  \item Present Simple for timetables;
  \item Future Continuous, Future Perfect, Future Perfect Continuous.
\end{itemize}

Later chapters will describe each future tense in more detail.

%=========================================================
\chapter{Future Simple (will) and Going To}
%=========================================================

\section{Future Simple with \enquote{will}}

\subsection{Form}

\begin{center}
Subject + will + base form
\end{center}

Examples:

\begin{itemize}
  \item I will call you later.
  \item She will help you with your homework.
\end{itemize}

\subsection{Uses}

\begin{itemize}
  \item decisions made at the moment of speaking;
  \item predictions based on opinion;
  \item promises and offers.
\end{itemize}

\section{Be Going To}

\subsection{Form}

\begin{center}
Subject + am / is / are + going to + base form
\end{center}

Examples:

\begin{itemize}
  \item I am going to study tonight.
  \item They are going to visit their grandparents.
\end{itemize}

\subsection{Uses}

\begin{itemize}
  \item intentions decided before speaking;
  \item predictions based on present evidence:
    \begin{itemize}
      \item Look at those clouds. It is going to rain.
    \end{itemize}
\end{itemize}

%=========================================================
\chapter{Future Continuous, Perfect and Perfect Continuous}
%=========================================================

\section{Future Continuous}

\subsection{Form}

\begin{center}
Subject + will be + verb-ing
\end{center}

Example:

\begin{itemize}
  \item This time tomorrow, I will be flying to London.
\end{itemize}

\section{Future Perfect Simple}

\subsection{Form}

\begin{center}
Subject + will have + past participle
\end{center}

Example:

\begin{itemize}
  \item By Friday, I will have finished my report.
\end{itemize}

\section{Future Perfect Continuous}

\subsection{Form}

\begin{center}
Subject + will have been + verb-ing
\end{center}

Example:

\begin{itemize}
  \item By next year, I will have been working here for ten years.
\end{itemize}

%=========================================================
\chapter{Passive Voice in Detail}
%=========================================================

\section{Active vs Passive}

Active voice focuses on the doer of the action:

\begin{quote}
The teacher explains the lesson.
\end{quote}

Passive voice focuses on the receiver of the action:

\begin{quote}
The lesson is explained by the teacher.
\end{quote}

\section{General Pattern}

\begin{center}
BE (in the correct tense) + past participle
\end{center}

\section{Passive Forms Across Tenses}

\begin{center}
\begin{tabular}{lll}
\toprule
Tense & Active & Passive \\
\midrule
Present Simple & They clean the room. & The room is cleaned. \\
Present Continuous & They are cleaning the room. & The room is being cleaned. \\
Present Perfect & They have cleaned the room. & The room has been cleaned. \\
Past Simple & They cleaned the room. & The room was cleaned. \\
Past Continuous & They were cleaning the room. & The room was being cleaned. \\
Past Perfect & They had cleaned the room. & The room had been cleaned. \\
Future Simple & They will clean the room. & The room will be cleaned. \\
Future Perfect & They will have cleaned it. & It will have been cleaned. \\
\bottomrule
\end{tabular}
\end{center}

%=========================================================
\chapter{Conditionals and Time Clauses}
%=========================================================

\section{Zero, First, Second and Third Conditional}

\begin{center}
\begin{tabular}{lll}
\toprule
Type & If-clause & Main clause \\
\midrule
Zero  & Present Simple & Present Simple \\
First & Present Simple & will + base form \\
Second& Past Simple & would + base form \\
Third & Past Perfect & would have + past participle \\
\bottomrule
\end{tabular}
\end{center}

Examples:

\begin{itemize}
  \item If you heat water, it boils. (Zero: general truth)
  \item If it rains, I will stay home. (First: real future)
  \item If I had more time, I would travel more. (Second: unreal present)
  \item If I had studied, I would have passed the exam. (Third: unreal past)
\end{itemize}

\section{Time Clauses with When, After, Before, Until}

In time clauses about the future, English often uses the Present Simple:

\begin{itemize}
  \item When I arrive, I will call you.
  \item As soon as she finishes, she will send the report.
\end{itemize}

%=========================================================
\chapter{Summary Tables and Study Plan}
%=========================================================

\section{Summary of Simple Tenses}

\begin{center}
\begin{tabular}{lll}
\toprule
Tense & Form & Main Use \\
\midrule
Present Simple & do/does + base & habits, general truths \\
Past Simple & did + base & completed past actions \\
Future Simple & will + base & decisions, opinions, promises \\
\bottomrule
\end{tabular}
\end{center}

\section{Summary of Continuous Tenses}

\begin{center}
\begin{tabular}{lll}
\toprule
Tense & Form & Main Idea \\
\midrule
Present Continuous & am/is/are + -ing & action now / temporary \\
Past Continuous & was/were + -ing & action in progress in the past \\
Future Continuous & will be + -ing & action in progress in the future \\
\bottomrule
\end{tabular}
\end{center}

\section{Summary of Perfect Tenses}

\begin{center}
\begin{tabular}{lll}
\toprule
Tense & Form & Time Link \\
\midrule
Present Perfect & have/has + pp & past linked to present \\
Past Perfect & had + pp & past before past \\
Future Perfect & will have + pp & finished before a future time \\
\bottomrule
\end{tabular}
\end{center}

\section{Study Plan}

\begin{itemize}
  \item \textbf{Beginner}: Present Simple, Present Continuous, Past Simple,
        Future with \enquote{will} and \enquote{going to}.
  \item \textbf{Intermediate}: all present and past tenses, basic conditionals.
  \item \textbf{Advanced}: all perfect and perfect continuous tenses,
        passive forms, conditionals 0, 1, 2, 3.
\end{itemize}

%=========================================================
\chapter{Word Stress: The Rhythm of English}
%=========================================================

\section{Importance of Word Stress}

In English, we do not say each syllable with the same force. In one word,
there is always one syllable that is louder, longer, and higher in pitch
than the others. This is called \textbf{word stress}.

Understanding stress is crucial because:
\begin{itemize}
  \item It helps with listening comprehension;
  \item It makes your speech sound more natural;
  \item It can change the meaning of a word (e.g., \textit{REcord} vs \textit{reCORD}).
\end{itemize}

\section{Basic Rules of Word Stress}

While there are many exceptions, these basic rules help in most cases:

\begin{enumerate}
  \item \textbf{One word has only one main stress.}
  \item \textbf{We only stress vowels, not consonants.}
\end{enumerate}

\section{Stress According to Word Class}

\subsection{Two-Syllable Nouns and Adjectives}
Most two-syllable nouns and adjectives have the stress on the \textbf{first} syllable.

\begin{itemize}
  \item \textbf{TA}-ble, \textbf{PRES}-ent, \textbf{CLEV}-er, \textbf{HAP}-py.
\end{itemize}

\subsection{Two-Syllable Verbs}
Most two-syllable verbs have the stress on the \textbf{second} syllable.

\begin{itemize}
  \item de-\textbf{CIDE}, be-\textbf{GIN}, pre-\textbf{SENT}, re-\textbf{CORD}.
\end{itemize}

\section{Noun-Verb Pairs (Heteronyms)}

Some words change their stress depending on whether they are used as a
noun or a verb.

\begin{center}
\begin{tabular}{lll}
\toprule
Word & Noun (Stress on 1st) & Verb (Stress on 2nd) \\
\midrule
Object  & \textbf{OB}-ject (a thing) & ob-\textbf{JECT} (to disagree) \\
Present & \textbf{PRES}-ent (a gift) & pre-\textbf{SENT} (to give) \\
Record  & \textbf{RE}-cord (a disk)  & re-\textbf{CORD} (to store sound) \\
Desert  & \textbf{DE}-sert (sand)    & de-\textbf{SERT} (to abandon) \\
\bottomrule
\end{tabular}
\end{center}

\section{Stress With Suffixes}

\subsection{Suffixes that do not change stress}
Most common suffixes (\textit{-ed, -ing, -er, -es, -ly, -ness}) do not
change the stress of the base word.
\begin{itemize}
  \item \textbf{HAP}-py $\rightarrow$ \textbf{HAP}-pi-ly.
  \item \textbf{WORK} $\rightarrow$ \textbf{WORK}-er.
\end{itemize}

\subsection{Suffixes that attract stress}
Some suffixes always take the stress themselves.
\begin{itemize}
  \item \textit{-ee}: em-ploy-\textbf{EE}, ref-u-\textbf{GEE}.
  \item \textit{-ese}: Chi-\textbf{NESE}, Por-tu-\textbf{GUESE}.
  \item \textit{-ique}: u-\textbf{NIQUE}, an-\textbf{TIQUE}.
\end{itemize}

\section{Compound Words}

\begin{itemize}
  \item \textbf{Compound Nouns}: Stress is usually on the \textbf{first} word.\\
    Example: \textbf{GREEN}-house, \textbf{BLACK}-bird.
  \item \textbf{Compound Verbs/Adjectives}: Stress is usually on the \textbf{second} word.\\
    Example: to over-\textbf{FLOW}, old-\textbf{FASH}-ioned.
\end{itemize}

%=========================================================
\chapter{Storytelling: Connecting the Tenses}
%=========================================================

\section{The Narrative Framework}

When we tell a story, we use a combination of different past tenses to
organise information and guide the listener. This is often called
\textbf{Narrative Tenses}.

\section{The Roles of Different Tenses}

\begin{itemize}
  \item \textbf{Past Simple}: For the main events of the story (the \enquote{backbone}).
  \item \textbf{Past Continuous}: For background scenes and actions in progress.
  \item \textbf{Past Perfect}: For events that happened \textit{before} the main story starts.
  \item \textbf{Present Simple/Continuous}: Can be used for \enquote{dramatic effect}
        (the historic present).
\end{itemize}

\section{Structuring Your Story}

A good story often follows this structure:

\begin{enumerate}
    \item \textbf{The Setting}: Use Past Continuous and Past Perfect.
    \item \textbf{The Inciting Incident}: Use Past Simple.
    \item \textbf{The Action}: Use Past Simple.
    \item \textbf{The Resolution}: Use Past Simple.
\end{enumerate}

\section{Linkers and Transitions}

To keep the story flowing, use time expressions:

\begin{itemize}
  \item \textbf{To start}: Once upon a time, It all began when...
  \item \textbf{To show sequence}: Then, after that, subsequently, later on.
  \item \textbf{To show simultaneous actions}: While, as, meanwhile.
  \item \textbf{To show surprise}: Suddenly, all of a sudden, out of the blue.
\end{itemize}

\section{A Practical Example: The Lost Cabin}

\begin{displayquote}
It \textbf{was raining} heavily (Past Continuous - background). I \textbf{had been walking}
for three hours (Past Perfect Continuous - duration before) and I \textbf{had lost}
my map (Past Perfect - event before).

Suddenly, I \textbf{saw} a small cabin (Past Simple - main event). I \textbf{approached}
the door and \textbf{knocked} (Past Simple - sequence). A man \textbf{opened} it.
He \textbf{was wearing} a strange hat...
\end{displayquote}

\section{Practice Table: Transforming Narrative}

\begin{center}
\begin{tabular}{ll}
\toprule
Basic Fact (Past Simple) & Narrative Enrichment \\
\midrule
I saw a dog. & I was walking home when I saw a dog. \\
The bus left. & I arrived at the station, but the bus had already left. \\
It rained. & It had been raining all day, so the ground was wet. \\
\bottomrule
\end{tabular}
\end{center}

%=========================================================
\chapter{Deep Dive into Conditionals and \enquote{If} Clauses}
%=========================================================

\section{Beyond the Basics}

While the Zero, First, Second, and Third conditionals cover many situations,
advanced English uses more complex structures to express nuances of
possibility, regret, and conditionality.

\section{Mixed Conditionals}

Mixed conditionals combine elements of the second and third conditionals
when the time in the \enquote{if} clause is different from the time in the main
clause.

\subsection{Type A: Past Action with Present Result}
The condition is in the past, but the result is in the present.
\begin{center}
If + Past Perfect, + would + base form
\end{center}
\textit{Example}: If I \textbf{had taken} that job (past), I \textbf{would be}
rich now (present).

\subsection{Type B: Present Condition with Past Result}
A permanent state or present situation influenced a past event.
\begin{center}
If + Past Simple, + would have + past participle
\end{center}
\textit{Example}: If I \textbf{were} taller (always), I \textbf{would have played}
professional basketball in college (past).

\section{Alternatives to \enquote{If}}

You can use other words to express conditions, often adding specific
emphasis.

\begin{itemize}
  \item \textbf{Unless} (if not): \enquote{I won't go \textbf{unless} you come with me.}
  \item \textbf{Provided that / Providing}: (Very formal \enquote{if}): \enquote{You can stay
        \textbf{provided that} you keep quiet.}
  \item \textbf{As long as / So long as}: (Focus on the duration or strength of condition):
        \enquote{\textbf{As long as} you work hard, you will succeed.}
  \item \textbf{Suppose / Supposing}: (Often used for imagining): \enquote{\textbf{Suppose}
        you won the lottery, what would you do?}
\end{itemize}

\section{Inversion in Conditional Sentences}

In formal or literary English, we can drop \enquote{if} and invert the subject
and auxiliary verb.

\begin{itemize}
  \item \textbf{First Conditional}: \textit{Should you need} any help, please call me.
        (instead of \textit{If you should need...})
  \item \textbf{Second Conditional}: \textit{Were I} in your position, I would resign.
        (instead of \textit{If I were...})
  \item \textbf{Third Conditional}: \textit{Had I known} about the traffic, I would
        have left earlier. (instead of \textit{If I had known...})
\end{itemize}

\section{Conditionals with \enquote{Will} and \enquote{Would}}

Normally, we don't use \enquote{will} or \enquote{would} in the if-clause. However,
there are exceptions for \textbf{politeness} or \textbf{insisting}.

\begin{itemize}
  \item \enquote{If you \textbf{will} wait a moment, I'll see if he's free.} (Polite request)
  \item \enquote{If you \textbf{will} keep making noise, you'll have to leave.} (Insisting/Habit)
\end{itemize}

%=========================================================
\chapter{Nuances: Continuity, Habit and Change}
%=========================================================

\section{Expressing Continuity and Change}

English uses specific adverbs and structures to describe how a situation
evolves over time.

\section{Still, No Longer, Any More}

\begin{itemize}
  \item \textbf{Still}: Refers to a situation that continues. It usually goes in
        the middle of the sentence.
        \begin{itemize}
            \item \enquote{It is 11 p.m. and he is \textbf{still} working.}
        \end{itemize}
  \item \textbf{No longer}: Refers to a situation that has jumped from \enquote{yes}
        to \enquote{no}. It is formal.
        \begin{itemize}
            \item \enquote{He \textbf{no longer} lives here.}
        \end{itemize}
  \item \textbf{Any more / Any longer}: Have the same meaning as \textit{no longer}
        but are used with a negative verb at the \textbf{end} of the sentence.
        \begin{itemize}
            \item \enquote{He doesn't live here \textbf{any more}.}
            \item \enquote{I cannot wait \textbf{any longer}.}
        \end{itemize}
\end{itemize}

\section{Used to, Be used to, Get used to}

These three structures are often confused but have very different
grammatical and logical meanings.

\subsection{Used to + Infinitive}
Describes a \textbf{past habit or state} that is no longer true.
\begin{itemize}
  \item \textit{Meaning}: \enquote{I did it in the past, but I don't now.}
  \item \textit{Example}: I \textbf{used to smoke}, but I quit last year.
  \item \textit{Note}: It has no present form (use Present Simple for current habits).
\end{itemize}

\subsection{Be used to + -ing / noun}
Describes a state of \textbf{familiarity}.
\begin{itemize}
  \item \textit{Meaning}: \enquote{This is normal/familiar for me.}
  \item \textit{Example}: I \textbf{am used to} the cold weather because I live in Canada.
  \item \textit{Grammar}: Here, \enquote{to} is a preposition, so it must be
        followed by a noun or an -ing form.
\end{itemize}

\subsection{Get used to + -ing / noun}
Describes the \textbf{process} of becoming familiar.
\begin{itemize}
  \item \textit{Meaning}: \enquote{I am becoming familiar with this.}
  \item \textit{Example}: It was difficult at first, but I am \textbf{getting used to}
        waking up early.
\end{itemize}

\section{Summary Table of Habits and States}

\begin{center}
\begin{tabular}{lll}
\toprule
Structure & Grammar & Concept \\
\midrule
Used to & + base form & Past habit (finished) \\
Be used to & + -ing / noun & Familiarity (state) \\
Get used to& + -ing / noun & Becoming familiar (process) \\
\bottomrule
\end{tabular}
\end{center}

%=========================================================
\chapter{Connecting Ideas: Relative Pronouns and Clauses}
%=========================================================

\section{The Role of Relative Pronouns}

Relative pronouns are used to connect a clause or phrase to a noun or
pronoun. They help us provide more information about a person or thing
without starting a new sentence.

\section{Core Relative Pronouns}

\begin{itemize}
  \item \textbf{Who}: For people (subject). \textit{The woman \textbf{who} lives next door is a doctor.}
  \item \textbf{Whom}: For people (object - formal). \textit{The person \textbf{whom} I called was out.}
  \item \textbf{Whose}: For possession. \textit{The boy \textbf{whose} dog ran away is crying.}
  \item \textbf{Which}: For things and animals. \textit{The car \textbf{which} he bought is expensive.}
  \item \textbf{That}: For people and things (informal / defining). \textit{The book \textbf{that} I read was great.}
\end{itemize}

\section{Relative Adverbs}

We can also use these adverbs to link clauses:
\begin{itemize}
  \item \textbf{Where}: For places. \textit{That is the school \textbf{where} I studied.}
  \item \textbf{When}: For time. \textit{I remember the day \textbf{when} we first met.}
  \item \textbf{Why}: For reasons. \textit{I don't know the reason \textbf{why} she left.}
\end{itemize}

\section{Defining vs Non-defining Clauses}

This is a critical distinction in English grammar.

\subsection{Defining Relative Clauses}
These provide \textbf{essential} information. Without the clause, the
sentence doesn't make sense or changes meaning. No commas are used.
\begin{itemize}
    \item \enquote{The cyclists \textbf{who were wearing helmets} were safe.} (Only some cyclists).
\end{itemize}

\subsection{Non-defining Relative Clauses}
These provide \textbf{extra} information. The sentence makes sense
without them. We \textbf{must} use commas.
\begin{itemize}
    \item \enquote{My brother, \textbf{who lives in Paris}, is an architect.} (I have one brother; he lives in Paris).
    \item \textit{Note}: We cannot use \enquote{that} in non-defining clauses.
\end{itemize}

%=========================================================
\chapter{The Gerund and the Infinitive}
%=========================================================

\section{Introduction}

In English, when one verb follows another, the second verb is usually
either in the \textbf{Gerund} (-ing) or the \textbf{Infinitive} (to + base form).

\section{When to Use the Gerund (-ing)}

\begin{itemize}
  \item \textbf{As a subject}: \textbf{Swimming} is good exercise.
  \item \textbf{After prepositions}: He is good \textbf{at painting}.
  \item \textbf{After certain verbs}: enjoy, avoid, finish, suggest, mind, keep.
        \begin{itemize}
            \item \enquote{I \textbf{enjoy reading} science fiction.}
        \end{itemize}
\end{itemize}

\section{When to Use the Infinitive (to...)}

\begin{itemize}
  \item \textbf{To show purpose}: I went to the store \textbf{to buy} milk.
  \item \textbf{After adjectives}: It is \textbf{difficult to learn} Japanese.
  \item \textbf{After certain verbs}: want, hope, decide, agree, promise, refuse.
        \begin{itemize}
            \item \enquote{She \textbf{decided to leave} early.}
        \end{itemize}
\end{itemize}

\section{Verbs Followed by Both (Change in Meaning)}

Some verbs can be followed by both, but the meaning changes significantly.

\begin{center}
\begin{tabular}{lp{5cm}p{5cm}}
\toprule
Verb & + Gerund (-ing) & + Infinitive (to...) \\
\midrule
Remember & Recall a past memory. & Don't forget a future task. \\
         & \textit{I remember locking the door.} & \textit{Remember to lock the door.} \\
\midrule
Stop     & Quit an activity entirely. & Interrupt an action to do something else. \\
         & \textit{He stopped smoking.} & \textit{He stopped to smoke.} \\
\midrule
Try      & Experiment with a method. & Make an effort to do something difficult. \\
         & \textit{Try adding some salt.} & \textit{I am trying to move this sofa.} \\
\bottomrule
\end{tabular}
\end{center}

%=========================================================
\chapter{The Subjunctive Mood}
%=========================================================

\section{What is the Subjunctive?}

The Subjunctive is a special \enquote{mood} used to express \textbf{wishes},
\textbf{suggestions}, \textbf{demands}, or \textbf{hypothetical} situations.
It is more formal than the indicative mood.

\section{The Mandative Subjunctive}

Used after verbs or adjectives of \enquote{insisting} or \enquote{importance} (suggest,
recommend, demand, essential, vital). In this form, we use the \textbf{base form}
of the verb (no \enquote{-s} for 3rd person).

\begin{itemize}
  \item \enquote{The doctor recommended that he \textbf{rest} for a week.} (Not \textit{rests}).
  \item \enquote{It is essential that everyone \textbf{be} on time.} (Not \textit{is}).
  \item \enquote{I suggest she \textbf{take} a taxi.} (Not \textit{takes}).
\end{itemize}

\section{The Subjunctive with \enquote{Were}}

Used for unreal or hypothetical conditions (often after \enquote{if} or \enquote{I wish}).
We use \enquote{were} for all persons (I, you, he, she, it).

\begin{itemize}
  \item \enquote{If I \textbf{were} you, I would apologize.}
  \item \enquote{I wish he \textbf{were} here right now.}
  \item \enquote{If she \textbf{were} more careful, she wouldn't make mistakes.}
\end{itemize}

\section{Fixed Expressions}

Some fixed phrases in English use the subjunctive:
\begin{itemize}
  \item \textbf{God save} the King.
  \item \textbf{Long live} the President.
  \item \textbf{Heaven forbid}.
  \item \textbf{So be it}.
\end{itemize}

%=========================================================
\chapter{Comprehensive Guide: Active and Passive Voice}
%=========================================================

\section{Introduction to the Passive Voice}

The passive voice is one of the most important structures in English. We
use it when the \textbf{action} is more important than the person who
performs it, or when we do not know who performed the action.

\section{The General Transformation Rule}

To change a sentence from active to passive:
\begin{enumerate}
    \item The object of the active sentence becomes the subject of the passive.
    \item The verb changes: \textbf{BE (in the same tense as the active) + Past Participle}.
    \item The subject of the active sentence becomes the \enquote{by-agent} (optional).
\end{enumerate}

\section{Passive Voice Across All Tenses}

Here is a complete reference of how every tense transforms into the
passive voice.

\subsection{Present Tenses}

\begin{itemize}
  \item \textbf{Present Simple} (\textit{am/is/are + pp})
    \begin{itemize}
        \item Active: Scientists study these atoms.
        \item Passive: These atoms \textbf{are studied} by scientists.
    \end{itemize}
  \item \textbf{Present Continuous} (\textit{am/is/are + being + pp})
    \begin{itemize}
        \item Active: They are building a new stadium.
        \item Passive: A new stadium \textbf{is being built}.
    \end{itemize}
  \item \textbf{Present Perfect} (\textit{have/has + been + pp})
    \begin{itemize}
        \item Active: Someone has stolen my laptop.
        \item Passive: My laptop \textbf{has been stolen}.
    \end{itemize}
\end{itemize}

\subsection{Past Tenses}

\begin{itemize}
  \item \textbf{Past Simple} (\textit{was/were + pp})
    \begin{itemize}
        \item Active: Shakespeare wrote \enquote{Hamlet}.
        \item Passive: \enquote{Hamlet} \textbf{was written} by Shakespeare.
    \end{itemize}
  \item \textbf{Past Continuous} (\textit{was/were + being + pp})
    \begin{itemize}
        \item Active: The chef was preparing the meal.
        \item Passive: The meal \textbf{was being prepared} by the chef.
    \end{itemize}
  \item \textbf{Past Perfect} (\textit{had + been + pp})
    \begin{itemize}
        \item Active: They had finished the bridge before the storm.
        \item Passive: The bridge \textbf{had been finished} before the storm.
    \end{itemize}
\end{itemize}

\subsection{Future Tenses}

\begin{itemize}
  \item \textbf{Future Simple} (\textit{will + be + pp})
    \begin{itemize}
        \item Active: The company will launch a new product.
        \item Passive: A new product \textbf{will be launched} by the company.
    \end{itemize}
  \item \textbf{Future Perfect} (\textit{will + have + been + pp})
    \begin{itemize}
        \item Active: They will have completed the project by 2030.
        \item Passive: The project \textbf{will have been completed} by 2030.
    \end{itemize}
\end{itemize}

\section{Passive with Modal Verbs}

When using modals (\textit{can, must, should, may}), the pattern is:
\textbf{Modal + be + Past Participle}.

\begin{center}
\begin{tabular}{ll}
\toprule
Active & Passive \\
\midrule
You \textbf{can see} stars. & Stars \textbf{can be seen}. \\
You \textbf{must pay} the bill. & The bill \textbf{must be paid}. \\
They \textbf{should solve} this. & This \textbf{should be solved}. \\
We \textbf{might finish} it. & It \textbf{might be finished}. \\
\bottomrule
\end{tabular}
\end{center}

\section{Advanced Passive Structures}

\subsection{Verbs with Two Objects}
Some verbs (give, send, show, buy) have two objects. We can make two
passive sentences.
\begin{itemize}
    \item Active: They gave \textbf{him} (1) \textbf{a medal} (2).
    \item Passive 1: \textbf{He} was given a medal. (More common)
    \item Passive 2: \textbf{A medal} was given to him.
\end{itemize}

\subsection{The Impersonal Passive}
Used with verbs of perception/opinion (say, believe, think, report).
\begin{itemize}
    \item \textit{It is said that...}
    \item \textit{It is believed that...}
    \item Example: \textbf{It is believed that} the city was founded in 500 BC.
\end{itemize}

\section{Summary: Transformation Table}

\begin{center}
\begin{tabular}{lll}
\toprule
Tense & Active Verb & Passive Verb \\
\midrule
Present Simple & works & is worked \\
Present Continuous & is working & is being worked \\
Present Perfect & has worked & has been worked \\
Past Simple & worked & was worked \\
Past Continuous & was working & was being worked \\
Past Perfect & had worked & had been worked \\
Future Simple & will work & will be worked \\
Future Perfect & will have worked & will have been worked \\
Modals & must work & must be worked \\
\bottomrule
\end{tabular}
\end{center}

%=========================================================
\chapter{Connecting Ideas: Coordinating Conjunctions (FANBOYS)}
%=========================================================

\section{What are Coordinating Conjunctions?}

Coordinating conjunctions are words that connect words, phrases, or clauses
of equal grammatical importance. The easiest way to remember them is
through the acronym \textbf{FANBOYS}.

\section{The FANBOYS Break Down}

\begin{itemize}
  \item \textbf{For}: Explains reason or purpose (similar to \enquote{because}, but more formal).
        \textit{I went to bed early, \textbf{for} I was exhausted.}
  \item \textbf{And}: Adds one thing to another.
        \textit{She likes tea \textbf{and} coffee.}
  \item \textbf{Nor}: Used to present a second negative idea.
        \textit{He doesn't like math, \textbf{nor} does he like physics.}
  \item \textbf{But}: Shows contrast.
        \textit{I wanted to go, \textbf{but} I was too busy.}
  \item \textbf{Or}: Presents an alternative or choice.
        \textit{Do you want tea \textbf{or} coffee?}
  \item \textbf{Yet}: Shows contrast or surprise (similar to \enquote{but}).
        \textit{The weather was cold, \textbf{yet} we enjoyed our walk.}
  \item \textbf{So}: Shows a consequence or result.
        \textit{It was raining, \textbf{so} I took an umbrella.}
\end{itemize}

\section{Punctuation with Conjunctions}

When a coordinating conjunction connects two \textbf{independent clauses}
(sentences that could stand alone), we usually place a comma \textbf{before}
the conjunction.

\begin{itemize}
    \item \enquote{I studied hard. I passed the test.} $\rightarrow$
    \item \enquote{I studied hard\textbf{,} and I passed the test.}
\end{itemize}

However, if you are just connecting two words or short phrases, no comma is
needed.
\begin{itemize}
    \item \enquote{I ate an apple and a banana.} (No comma).
\end{itemize}

%=========================================================
\chapter{Direct and Indirect Questions}
%=========================================================

\section{Direct Questions}

Direct questions are the standard way of asking for information. They end
with a question mark and often use inversion (verb before subject).

\subsection{Yes/No Questions}
\begin{center}
Auxiliary + Subject + Main Verb?
\end{center}
\textit{Example}: \enquote{\textbf{Do you} speak English?}

\subsection{Wh- Questions}
\begin{center}
Wh- word + Auxiliary + Subject + Main Verb?
\end{center}
\textit{Example}: \enquote{\textbf{Where does} he live?}

\section{Indirect Questions}

Indirect questions are more polite and formal. They are often embedded
within another sentence or introductory phrase. \textbf{CRITICAL RULE}: In indirect
questions, there is \textbf{no inversion}. The word order is the same as a
normal statement (Subject + Verb).

\section{Forming Indirect Questions}

\subsection{Wh- Indirect Questions}
Introductory phrase + Wh- word + \textbf{Subject + Verb}
\begin{itemize}
  \item Direct: Where is the station?
  \item Indirect: \enquote{Could you tell me \textbf{where the station is}?}
  \item Direct: What time does the train leave?
  \item Indirect: \enquote{Do you know \textbf{what time the train leaves}?}
\end{itemize}

\subsection{Yes/No Indirect Questions}
Introductory phrase + \textbf{if / whether} + \textbf{Subject + Verb}
\begin{itemize}
  \item Direct: Does he like spicy food?
  \item Indirect: \enquote{I wonder \textbf{if he likes} spicy food.}
  \item Direct: Will it rain tomorrow?
  \item Indirect: \enquote{Do you know \textbf{whether it will rain} tomorrow?}
\end{itemize}

\section{Common Introductory Phrases}

\begin{itemize}
  \item Could you tell me...?
  \item Do you know...?
  \item I was wondering...?
  \item I'd like to know...?
  \item Could you explain...?
\end{itemize}

\section{Summary Table: Direct vs Indirect}

\begin{center}
\begin{tabular}{lll}
\toprule
Direct Question & Indirect Question & Change \\
\midrule
Where is he? & Do you know where he is? & No inversion \\
What did she say? & I'd like to know what she said. & No \enquote{did} + past tense \\
Is it cold? & I wonder if it is cold. & Use \enquote{if} \\
Can you help? & Could you tell me if you can help?& Use \enquote{if} \\
\bottomrule
\end{tabular}
\end{center}

%=========================================================
\chapter{Mastering Phrasal Verbs}
%=========================================================

\section{Introduction to Phrasal Verbs}

A phrasal verb is a combination of a \textbf{standard verb} and a
\textbf{particle} (an adverb or a preposition). Together, they create a
completely new meaning.

Example: \textbf{Look} (to see) vs \textbf{Look after} (to take care of).

\section{Grammatical Types of Phrasal Verbs}

\subsection{Intransitive (No Object)}
These verbs do not need an object to make sense.
\begin{itemize}
  \item \enquote{The car \textbf{broke down} on the highway.}
  \item \enquote{I \textbf{wake up} at 7 a.m.}
\end{itemize}

\subsection{Transitive (Need an Object)}
These can be either separable or inseparable.

\subsubsection{Separable}
The object can go after the particle or between the verb and the particle.
\begin{itemize}
  \item \enquote{I \textbf{turned off} the light.} OR \enquote{I \textbf{turned} the light \textbf{off}.}
  \item \textit{Note}: If the object is a \textbf{pronoun}, it MUST go in the middle.
    \begin{itemize}
        \item \enquote{I turned \textbf{it} off.} (Correct)
        \item \enquote{I turned off it.} (Wrong)
    \end{itemize}
\end{itemize}

\subsubsection{Inseparable}
The object MUST go after the particle.
\begin{itemize}
  \item \enquote{I am \textbf{looking for} my keys.} (Not: \textit{looking my keys for}).
  \item \enquote{He \textbf{ran into} an old friend.}
\end{itemize}

\section{Common Particles and Their Logic}

Sometimes, the particle gives a clue to the meaning:
\begin{itemize}
  \item \textbf{UP} (completing or increasing): drink up, eat up, heat up.
  \item \textbf{DOWN} (decreasing or stopping): slow down, calm down, cut down.
  \item \textbf{OFF} (leaving or stopping): take off, go off, turn off.
  \item \textbf{ON} (continuing): carry on, keep on, go on.
\end{itemize}

\section{Essential Phrasal Verbs Table}

\begin{center}
\begin{tabular}{lll}
\toprule
Verb & Meaning & Example \\
\midrule
Get along & Have a good relationship & I \textbf{get along} with my boss. \\
Give up & Stop trying / quit & Never \textbf{give up} on your dreams. \\
Put off & Postpone / delay & Don't \textbf{put off} your homework. \\
Bring up & Mention a topic & She \textbf{brought up} a new idea. \\
Take after & Resemble a parent & He \textbf{takes after} his father. \\
\bottomrule
\end{tabular}
\end{center}

%=========================================================
\chapter{Idiomatic Expressions and Figurative Language}
%=========================================================

\section{What are Idioms?}

An idiom is a group of words whose collective meaning is different from
the literal meaning of the individual words. They make your English sound
natural and \enquote{fluent}.

\section{Common Thematic Idioms}

\subsection{Expressions about Time}
\begin{itemize}
  \item \textbf{Once in a blue moon}: Very rarely.
        \enquote{I see him once in a blue moon.}
  \item \textbf{Beating around the bush}: Avoiding the main topic.
        \enquote{Stop beating around the bush and tell me the truth.}
\end{itemize}

\subsection{Expressions about Feelings and Health}
\begin{itemize}
  \item \textbf{Under the weather}: Feeling slightly ill.
        \enquote{I'm feeling a bit under the weather today.}
  \item \textbf{Over the moon}: Extremely happy.
        \enquote{She was over the moon when she passed the exam.}
\end{itemize}

\subsection{Expressions about Work and Effort}
\begin{itemize}
  \item \textbf{A piece of cake}: Something very easy.
        \enquote{The test was a piece of cake.}
  \item \textbf{Get the ball rolling}: To start a process.
        \enquote{Let's get the ball rolling on this project.}
\end{itemize}

\section{Why Idioms Matter}

Idioms are not just for \enquote{slang}. They are used in professional
settings, news, and daily conversation. However, you should use them
carefully:
\begin{enumerate}
    \item Don't over-use them;
    \item Make sure you understand the register (formal vs informal);
    \item Learn them in context.
\end{enumerate}

\section{Practice Table: Literal vs Idiomatic}

\begin{center}
\begin{tabular}{ll}
\toprule
Literal English & Idiomatic English \\
\midrule
It is raining heavily. & It's raining cats and dogs. \\
I am listening carefully. & I'm all ears. \\
He is very busy. & He has a lot on his plate. \\
Good luck. & Break a leg! \\
\bottomrule
\end{tabular}
\end{center}

%=========================================================
\chapter{Word Transformation: Building Your Vocabulary}
%=========================================================

\section{Introduction to Word Formation}

Word transformation (or word formation) is the process of creating new
words by adding \textbf{prefixes} or \textbf{suffixes} to a base word
(root). This is a powerful way to expand your vocabulary naturally.

\section{Common Prefixes and Their Meanings}

Prefixes are added to the \textbf{beginning} of a word. They often change
 the meaning to its opposite.

\begin{itemize}
  \item \textbf{Un- / In- / Im- / Ir- / Il-} (Not/Opposite):
        unhappy, invisible, impossible, irregular, illegal.
  \item \textbf{Re-} (Again): rewrite, redo, reappear.
  \item \textbf{Pre-} (Before): prevent, preschool, prehistoric.
  \item \textbf{Post-} (After): postpone, post-graduate.
  \item \textbf{Mis-} (Wrongly): misunderstand, mislead, mistake.
\end{itemize}

\section{Common Suffixes and Word Classes}

Suffixes are added to the \textbf{end} of a word and usually change its
grammatical class (e.g., from a verb to a noun).

\subsection{Creating Nouns}
\begin{itemize}
  \item \textbf{-ment}: develop $\rightarrow$ develop\textbf{ment}.
  \item \textbf{-tion / -ion}: operate $\rightarrow$ opera\textbf{tion}.
  \item \textbf{-ness}: happy $\rightarrow$ happi\textbf{ness}.
  \item \textbf{-ity}: able $\rightarrow$ abil\textbf{ity}.
  \item \textbf{-er / -or}: teach $\rightarrow$ teach\textbf{er}, act $\rightarrow$ act\textbf{or}.
\end{itemize}

\subsection{Creating Adjectives}
\begin{itemize}
  \item \textbf{-ful}: help $\rightarrow$ help\textbf{ful}.
  \item \textbf{-less}: care $\rightarrow$ care\textbf{less}.
  \item \textbf{-able / -ible}: comfort $\rightarrow$ comfort\textbf{able}.
  \item \textbf{-ive}: create $\rightarrow$ creat\textbf{ive}.
  \item \textbf{-ous}: danger $\rightarrow$ danger\textbf{ous}.
\end{itemize}

\subsection{Creating Verbs}
\begin{itemize}
  \item \textbf{-ise / -ize}: modern $\rightarrow$ modern\textbf{ise}.
  \item \textbf{-en}: sharp $\rightarrow$ sharp\textbf{en}.
\end{itemize}

\section{Summary Table: Transformation Patterns}

\begin{center}
\begin{tabular}{llll}
\toprule
Verb & Noun & Adjective & Adverb \\
\midrule
Succeed & Success & Successful & Successfully \\
Compete & Competition & Competitive & Competitively \\
Imagine & Imagination & Imaginative & Imaginatively \\
Care    & Care        & Careful     & Carefully \\
Differ  & Difference  & Different   & Differently \\
\bottomrule
\end{tabular}
\end{center}

%=========================================================
\chapter{Reported Speech: Indirect Communication}
%=========================================================

\section{What is Reported Speech?}

Reported speech (indirect speech) is used when we tell someone what another
person said, without using their exact words.

\section{The General Rule: Tense Backshift}

In Reported Speech, we usually move the tenses one step \enquote{back} into the
past if the reporting verb (said, told, etc.) is in the past.

\begin{center}
\begin{tabular}{ll}
\toprule
Direct Speech & Reported Speech \\
\midrule
Present Simple $\rightarrow$ & Past Simple \\
Present Continuous $\rightarrow$ & Past Continuous \\
Present Perfect $\rightarrow$ & Past Perfect \\
Past Simple $\rightarrow$ & Past Perfect \\
Will $\rightarrow$ & Would \\
Can $\rightarrow$ & Could \\
May $\rightarrow$ & Might \\
Must $\rightarrow$ & Had to \\
\bottomrule
\end{tabular}
\end{center}

\section{Changes in Pronouns and Determiners}

Pronouns change depending on who is speaking to whom.
\begin{itemize}
    \item Direct: \enquote{\textbf{I} like \textbf{my} car.}
    \item Reported: He said \textbf{he} liked \textbf{his} car.
\end{itemize}

\section{Changes in Time and Place Expressions}

\begin{itemize}
  \item \textbf{Now} $\rightarrow$ then / at that moment.
  \item \textbf{Today} $\rightarrow$ that day.
  \item \textbf{Yesterday} $\rightarrow$ the day before / the previous day.
  \item \textbf{Tomorrow} $\rightarrow$ the next day / the following day.
  \item \textbf{Here} $\rightarrow$ there.
  \item \textbf{This} $\rightarrow$ that.
\end{itemize}

\section{Reporting Questions}

\subsection{Wh- Questions}
Keep the Wh- word but use \textbf{statement word order} (no inversion).
\begin{itemize}
    \item Direct: \enquote{Where are you going?}
    \item Reported: She asked me \textbf{where I was going}.
\end{itemize}

\subsection{Yes/No Questions}
Use \textbf{if} or \textbf{whether} and statement word order.
\begin{itemize}
    \item Direct: \enquote{Do you like coffee?}
    \item Reported: He asked me \textbf{if I liked} coffee.
\end{itemize}

\section{Reporting Commands and Requests}

Use the pattern: \textbf{Verb + Object + to-infinitive}.
\begin{itemize}
  \item Direct: \enquote{Sit down!}
  \item Reported: The teacher told them \textbf{to sit down}.
  \item Direct: \enquote{Please help me.}
  \item Reported: She asked me \textbf{to help her}.
\end{itemize}

%=========================================================
\chapter{Paired Connectors: Correlative Conjunctions}
%=========================================================

\section{What are Correlative Conjunctions?}

Correlative conjunctions are pairs of words that work together to connect
elements in a sentence. They always travel in \textbf{pairs}.

\section{The Primary Pairs}

\begin{itemize}
  \item \textbf{Both ... and}: Used to emphasize two related ideas.
        \textit{\textbf{Both} the teacher \textbf{and} the students were happy.}
  \item \textbf{Either ... or}: Used to present a choice between two options.
        \textit{You can \textbf{either} stay here \textbf{or} come with us.}
  \item \textbf{Neither ... nor}: Used to join two negative ideas.
        \textit{\textbf{Neither} his father \textbf{nor} his mother could attend.}
  \item \textbf{Not only ... but also}: Used to add extra, often surprising, information.
        \textit{She is \textbf{not only} talented \textbf{but also} very hard-working.}
  \item \textbf{Whether ... or}: Used to express doubt or choice between two possibilities.
        \textit{I don't know \textbf{whether} he is coming \textbf{or} not.}
\end{itemize}

\section{Critical Rule 1: Parallel Structure}

When using correlative conjunctions, the elements being connected must be
grammatically \textbf{parallel}.

\begin{itemize}
  \item \textbf{Correct}: \textit{He likes \textbf{both} swimming \textbf{and} running.} (Two gerunds).
  \item \textbf{Incorrect}: \textit{He likes \textbf{both} swimming \textbf{and} to run.} (Gerund and infinitive).
\end{itemize}

\section{Critical Rule 2: Subject-Verb Agreement}

When connecting two subjects with \textbf{either...or} or \textbf{neither...nor},
the verb must agree with the \textbf{closer subject}.

\begin{itemize}
  \item \enquote{Neither the teacher nor the \textbf{students are} here.} (Students is closer, so plural verb).
  \item \enquote{Either the students or the \textbf{teacher is} here.} (Teacher is closer, so singular verb).
  \item \textit{Note}: With \textbf{both...and}, the verb is \textbf{always plural}.
    \begin{itemize}
        \item \enquote{\textbf{Both} Mike \textbf{and} Sarah \textbf{are} coming.}
    \end{itemize}
\end{itemize}

\section{Summary: Correlative Conjunctions Usage}

\begin{center}
\begin{tabular}{lll}
\toprule
Pair & Meaning & Example \\
\midrule
Both...and & Addition (+) & Both tea and coffee are available. \\
Either...or & Choice (A/B) & You can either call or text. \\
Neither...nor & Negative (--) & Neither the bus nor the train came. \\
Not only...but also& Emphasis (++) & He's not only fast but also strong. \\
\bottomrule
\end{tabular}
\end{center}

\backmatter

\chapter*{Conclusion}
\addcontentsline{toc}{chapter}{Conclusion}

This book has presented the main English tenses in a systematic way.
To truly master them, you should now practise:

\begin{itemize}
  \item by writing your own sentences for each tense;
  \item by transforming sentences from one tense to another;
  \item by reading real texts and identifying the tenses used;
  \item by speaking and listening regularly in English.
\end{itemize}

Over time, accurate use of tenses will become natural and intuitive.

\end{document}
