\documentclass[a4paper,12pt]{article}
\usepackage[utf8]{inputenc}
\usepackage[T1]{fontenc}
\usepackage[french]{babel}
\usepackage{hyperref}
\usepackage{graphicx}
\usepackage{geometry}
\usepackage{xcolor}
\usepackage{fancyhdr}
\usepackage{titlesec}

% Configuration de la page
\geometry{hmargin=2.5cm,vmargin=2.5cm}
\hypersetup{
    colorlinks=true,
    linkcolor=blue,
    filecolor=magenta,      
    urlcolor=cyan,
}

% En-tête et pied de page
\pagestyle{fancy}
\fancyhf{}
\rhead{AYLF Group Tracker - Guide Utilisateur}
\lhead{\leftmark}
\cfoot{\thepage}

\title{\textbf{Guide Utilisateur Complet}\\AYLF Group Tracker}
\author{Équipe Technique AYLF}
\date{Janvier 2025 - Version 3.1}

\begin{document}

\maketitle
\thispagestyle{empty}

\begin{abstract}
Ce document est le guide de référence pour l'utilisation de l'application AYLF Group Tracker. Il couvre l'ensemble des fonctionnalités, y compris les dernières mises à jour concernant la gestion financière, les rapports, la sécurité, l'installation mobile (PWA) et les outils de diagnostic d'identité.
\end{abstract}

\tableofcontents
\newpage

\section{Introduction et Nouveautés}

Bienvenue dans la version **3.1** de l'application **AYLF Group Tracker**. Cette plateforme centralise la gestion des sites, groupes, activités et membres de l'AYLF.

\subsection{Dernières Améliorations (v3.1)}
Cette version intègre des fonctionnalités majeures pour garantir un état "Zero Gap" et une robustesse totale :
\begin{itemize}
    \item \textbf{Diagnostic d'Identité} : Outil d'auto-réparation pour synchroniser les comptes Kinde et les profils locaux.
    \item \textbf{Sécurité Hiérarchique (RLS)} : Protection 100\% granulaire des données (NC $\rightarrow$ Sites $\rightarrow$ Groupes).
    \item \textbf{Visibilité Financière Totale} : Accès aux indicateurs de performance (KPI) pour tous les niveaux administratifs.
    \item \textbf{Cascade d'Intégrité} : Mise à jour automatique des relations de données lors des synchronisations de profil.
    \item \textbf{Application Mobile (PWA)} : Installation directe sur smartphone pour un accès rapide.
\end{itemize}

\section{Diagnostic et Synchronisation d'Identité}
\label{sec:identity}

Pour garantir la sécurité, l'application vérifie que votre identifiant d'authentification (Kinde) correspond parfaitement à votre profil dans la base de données.

\subsection{Signes d'un Problème d'Identité}
Si vous voyez une bannière d'alerte rouge \textbf{"Identité non synchronisée"} ou si vous recevez une erreur \texttt{ID\_MISMATCH} lors de la création d'un membre ou d'une activité :
\begin{enumerate}
    \item Cliquez sur le lien \textbf{"Diagnostiquer le problème"} dans l'alerte.
    \item Ou naviguez vers \textbf{Paramètres > Diagnostics} dans le menu latéral.
\end{enumerate}

\subsection{Procédure de Synchronisation}
L'outil de diagnostic affiche l'état de votre session. Si un bouton \textbf{"Synchroniser mon identité"} apparaît :
\begin{enumerate}
    \item Cliquez sur le bouton.
    \item Le système mettra à jour votre profil local pour correspondre à votre compte Kinde.
    \item Grâce à la \textit{Cascade de Mise à jour}, toutes vos données liées (rapports, activités) resteront intactes.
    \item Une fois terminé, l'accès total sera rétabli.
\end{enumerate}

\section{Installation Mobile (PWA)}
\label{sec:pwa}

L'application est désormais une \textbf{Progressive Web App (PWA)}. Vous pouvez l'installer sur votre téléphone comme une application native.

\subsection{Sur Android (Chrome)}
\begin{enumerate}
    \item Ouvrez Chrome et naviguez vers l'application.
    \item Appuyez sur le menu (trois points) en haut à droite.
    \item Sélectionnez \textbf{"Installer l'application"} ou \textbf{"Ajouter à l'écran d'accueil"}.
    \item L'icône AYLF apparaîtra sur votre écran d'accueil.
\end{enumerate}

\subsection{Sur iOS (Safari)}
\begin{enumerate}
    \item Ouvrez Safari et naviguez vers l'application.
    \item Appuyez sur le bouton \textbf{Partager} (carré avec une flèche vers le haut).
    \item Faites défiler vers le bas et appuyez sur \textbf{"Sur l'écran d'accueil"}.
    \item Confirmez en appuyant sur \textbf{"Ajouter"}.
\end{enumerate}

\section{Gestion des Rapports et Finances}

\subsection{Cycle de Vie d'un Rapport}
Le processus de soumission et de validation des rapports a été enrichi :
\begin{enumerate}
    \item \textbf{Soumission} : Le Leader ou Coordinateur soumet un rapport d'activité.
    \item \textbf{Validation} : Le Coordinateur National examine le rapport.
    \item \textbf{Approbation} :
    \begin{itemize}
        \item Le statut passe à \texttt{Approved}.
        \item \textbf{Nouveau :} Si le rapport contient des dépenses, une \textbf{transaction financière de dépense} est automatiquement créée et liée au rapport.
    \end{itemize}
    \item \textbf{Rejet} :
    \begin{itemize}
        \item Le statut passe à \texttt{Rejected}.
        \item \textbf{Nouveau :} Un motif de rejet est obligatoire et s'affiche clairement sur la carte du rapport pour le soumissionnaire (encadré rouge).
    \end{itemize}
\end{enumerate}

\subsection{Suivi Financier}
Les finances sont désormais étroitement liées aux activités :
\begin{itemize}
    \item \textbf{Allocations} : Les fonds alloués aux sites/groupes sont suivis.
    \item \textbf{Dépenses Automatiques} : Chaque rapport approuvé génère une écriture comptable.
    \item \textbf{Transparence} : Les coordinateurs peuvent voir le solde en temps réel de leur entité.
\end{itemize}

\section{Rôles et Permissions}

L'application gère strictement les accès via 4 rôles, respectant la hiérarchie AYLF :

\begin{description}
    \item[Coordinateur National] Accès total à tous les sites et groupes du pays. Supervision financière et approbation des rapports.
    \item[Coordinateur de Site] Gestion de son site spécifique et de tous les "Small Groups" rattachés. Accès aux statistiques du site.
    \item[Leader de Small Group] Gestion exclusive de son propre groupe et de ses membres.
    \item[Membre] Consultation de son propre profil et des activités de son groupe.
\end{description}

\section{Sécurité et Intégrité}

\subsection{Protection des Données (RLS)}
Chaque opération de lecture ou d'écriture est vérifiée au niveau de la base de données (Row-Level Security). Un Leader ne peut jamais modifier les données d'un autre groupe, garantissant la confidentialité et l'intégrité de la hiérarchie.

\subsection{Traçabilité (Audit Logs)}
Toutes les actions sensibles (approbation de dépenses, suppression de membres) sont enregistrées avec l'identité de l'auteur, la date et l'adresse IP pour une transparence totale.

\end{document}
