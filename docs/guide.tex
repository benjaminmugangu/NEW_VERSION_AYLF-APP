\documentclass[a4paper,12pt]{article}
\usepackage[utf8]{inputenc}
\usepackage[T1]{fontenc}
\usepackage[french]{babel}
\usepackage{hyperref}
\usepackage{graphicx}
\usepackage{geometry}
\usepackage{xcolor}
\usepackage{fancyhdr}
\usepackage{titlesec}

% Configuration de la page
\geometry{hmargin=2.5cm,vmargin=2.5cm}
\hypersetup{
    colorlinks=true,
    linkcolor=blue,
    filecolor=magenta,      
    urlcolor=cyan,
}

% En-tête et pied de page
\pagestyle{fancy}
\fancyhf{}
\rhead{AYLF Group Tracker - Guide Utilisateur}
\lhead{\leftmark}
\cfoot{\thepage}

\title{\textbf{Guide Utilisateur Complet}\\AYLF Group Tracker}
\author{Équipe Technique AYLF}
\date{Décembre 2024 - Version 2.0}

\begin{document}

\maketitle
\thispagestyle{empty}

\begin{abstract}
Ce document est le guide de référence pour l'utilisation de l'application AYLF Group Tracker. Il couvre l'ensemble des fonctionnalités, y compris les dernières mises à jour concernant la gestion financière, les rapports, la sécurité et l'installation mobile (PWA).
\end{abstract}

\tableofcontents
\newpage

\section{Introduction et Nouveautés}

Bienvenue dans la nouvelle version de l'application **AYLF Group Tracker**. Cette plateforme centralise la gestion des sites, groupes, activités et membres de l'AYLF.

\subsection{Dernières Améliorations (v2.0)}
Cette version intègre des fonctionnalités majeures suite à l'audit de décembre 2024 :
\begin{itemize}
    \item \textbf{Application Mobile (PWA)} : Installation directe sur smartphone pour un accès hors ligne.
    \item \textbf{Gestion Financière Automatisée} : Génération automatique des transactions lors de l'approbation des rapports.
    \item \textbf{Feedback de Rejet} : Visualisation claire des motifs de rejet des rapports.
    \item \textbf{Sécurité Renforcée} : Protection des données et traçabilité accrue.
    \item \textbf{Interface Améliorée} : Mode sombre, chargement rapide et navigation fluide.
\end{itemize}

\section{Installation Mobile (PWA)}
\label{sec:pwa}

L'application est désormais une \textbf{Progressive Web App (PWA)}. Vous pouvez l'installer sur votre téléphone comme une application native.

\subsection{Sur Android (Chrome)}
\begin{enumerate}
    \item Ouvrez Chrome et naviguez vers l'application.
    \item Appuyez sur le menu (trois points) en haut à droite.
    \item Sélectionnez \textbf{"Installer l'application"} ou \textbf{"Ajouter à l'écran d'accueil"}.
    \item L'icône AYLF apparaîtra sur votre écran d'accueil.
\end{enumerate}

\subsection{Sur iOS (Safari)}
\begin{enumerate}
    \item Ouvrez Safari et naviguez vers l'application.
    \item Appuyez sur le bouton \textbf{Partager} (carré avec une flèche vers le haut).
    \item Faites défiler vers le bas et appuyez sur \textbf{"Sur l'écran d'accueil"}.
    \item Confirmez en appuyant sur \textbf{"Ajouter"}.
\end{enumerate}

\section{Gestion des Rapports et Finances}

\subsection{Cycle de Vie d'un Rapport}
Le processus de soumission et de validation des rapports a été enrichi :
\begin{enumerate}
    \item \textbf{Soumission} : Le Leader ou Coordinateur soumet un rapport d'activité.
    \item \textbf{Validation} : Le Coordinateur National examine le rapport.
    \item \textbf{Approbation} :
    \begin{itemize}
        \item Le statut passe à \texttt{Approved}.
        \item \textbf{Nouveau :} Si le rapport contient des dépenses, une \textbf{transaction financière de dépense} est automatiquement créée et liée au rapport.
    \end{itemize}
    \item \textbf{Rejet} :
    \begin{itemize}
        \item Le statut passe à \texttt{Rejected}.
        \item \textbf{Nouveau :} Un motif de rejet est obligatoire et s'affiche clairement sur la carte du rapport pour le soumissionnaire (encadré rouge).
    \end{itemize}
\end{enumerate}

\subsection{Suivi Financier}
Les finances sont désormais étroitement liées aux activités :
\begin{itemize}
    \item \textbf{Allocations} : Les fonds alloués aux sites/groupes sont suivis.
    \item \textbf{Dépenses Automatiques} : Chaque rapport approuvé génère une écriture comptable.
    \item \textbf{Transparence} : Les coordinateurs peuvent voir le solde en temps réel de leur entité.
\end{itemize}

\section{Rôles et Permissions (Rappel)}

L'application gère strictement les accès via 4 rôles :

\begin{description}
    \item[Coordinateur National] Accès total. Gestion des sites, utilisateurs, validation des rapports et supervision financière globale.
    \item[Coordinateur de Site] Gestion de son site uniquement. Création de groupes, supervision des leaders, rapports de site.
    \item[Leader de Small Group] Gestion de son groupe. Ajout de membres, rapports d'activités de groupe.
    \item[Membre] Consultation uniquement (profil, activités à venir).
\end{description}

\section{Sécurité et Performance}

\subsection{Sécurité des Données}
\begin{itemize}
    \item \textbf{Rate Limiting} : Protection contre les tentatives de connexion abusives.
    \item \textbf{Audit Logs} : Toutes les actions sensibles (approbation, suppression) sont tracées.
    \item \textbf{Headers de Sécurité} : Protection renforcée contre les attaques XSS et autres vulnérabilités web.
\end{itemize}

\subsection{Performance}
\begin{itemize}
    \item \textbf{Pagination} : Les listes chargent les données page par page pour plus de rapidité.
    \item \textbf{Mode Hors Ligne} : Grâce à la PWA, certaines pages restent accessibles sans connexion (cache).
\end{itemize}

\end{document}
